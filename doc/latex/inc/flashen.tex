

\chapter{Flashen}


\section{Linux}

\subsection{Vorbereitung}

STM stellt für den ST-Link/V2-Programmer keine Software für Linux zur Verfügung. Dieser Adapter kann aber dennoch zum Flashen über Linux benutzt werden, wenn das auf GitHub verfügbare Programm \texttt{stlink} verwendet wird. Dieses kann wie folgt installiert werden.\footnote{Vorgehen nach Nicolas Acero.}

\begin{enumerate}
	\item Für Installation benötigte Pakete installieren
		\begin{verbatim}
			sudo apt-get install libusb-1.0-0-dev git unzip libgtk-3-dev
		\end{verbatim}
	\item stlink herunterladen (hier Version 1.2.0, \ggf neuere Version nehmen)
		\begin{verbatim}
			wget https://github.com/texane/stlink/archive/1.2.0.zip
		\end{verbatim}
	\item Datei entpacken und in das Verzeichnis wechseln
		\begin{verbatim}
			# Unzip the file
			unzip 1.2.0.zip
			cd stlink-1.2.0
		\end{verbatim}
	\item Programm kompilieren
		\begin{verbatim}
			mkdir build && cd build
			cmake -DCMAKE_BUILD_TYPE=Debug ..
			make		
		\end{verbatim}
	\item stlink installieren
		\begin{verbatim}
			sudo make install
		\end{verbatim}
	\item udev-Regeln so einstellen, dass st-flash ohne "`sudo"' verwendet werden kann
		\begin{verbatim}
			cd ..
			sudo cp *.rules /etc/udev/rules.d
			sudo restart udev
		\end{verbatim}
	\item Prorgamm sollte sich jetzt öffnen lassen
		\begin{verbatim}
			stlink-gui
		\end{verbatim}
\end{enumerate}


\subsection{Flashen}


\begin{enumerate}
	\item ST-Link/V2 an Platine anschließen. Fahrzeug einschalten.
	\item Öffnen der grafischen Oberfläche von \texttt{stlink}
		\begin{verbatim}
			stlink-gui
		\end{verbatim}
	\item Auf "`Connect"'-Icon klicken. Es sollten die Daten des Mikrocontrollers angezeigt werden.
	\item Auf "`Open File"'-Icon klicken und bin-Datei auswählen.
		\begin{itemize}
			\item Im Repository unter \verb|fw_releases\| liegen die "`offiziellen"' Versionen.
		\end{itemize}
	\item Auf "`Program"'-Icon klicken.
		\begin{itemize}
			\item Start-Adresse ist ist \verb|0x08000000|. (Sollte voreingestellt sein.)
		\end{itemize}
\end{enumerate}


\section{Windows}

\subsection{Vorbereitung}

"`STM32 ST-Link Utility"' von der STM-Homepage herunterladen und installieren (beinhaltet auch die Treiber). 

\textbf{Hinweis:} Die könnte schon problematisch sein, wenn man den ST-Link/V2 zum Debuggen verwendet will, da damit auch ein spezieller Treiber installiert wird, für gbd aber ein generischer Treiber benötigt wird. (Quelle: Forum, quergelesen. Müsste noch geklärt werden.)



\subsection{Flashen}

\begin{enumerate}
	\item ST-Link/V2 an Platine anschließen. Fahrzeug einschalten.
	\item "`STM32 ST-Link/V2 Utility"' öffnen.
	\item Auf "`Connect"'-Icon (drittes von links) klicken. Es sollten die Daten des Mikrocontrollers angezeigt werden.
	\item Auf "`Open File"'-Icon (erstes von links) klicken und bin-Datei auswählen.
		\begin{itemize}
			\item Im Repository unter \verb|fw_releases\| liegen die "`offiziellen"' Versionen.
		\end{itemize}
	\item Auf "`Program Verify"'-Icon (sechstes von links) klicken.
		\begin{itemize}
			\item Start-Adresse ist ist \verb|0x08000000|. (Sollte voreingestellt sein.)
		\end{itemize}
\end{enumerate}



