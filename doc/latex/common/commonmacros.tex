% Inhalt
% ======
%	Selbstdefinierte Trennregeln
%	Allgemeine Abk�rzungen
%	Makros f�r Referenzen (Abbildungen, Zitate, ...)
%	Makros f�r Abbildungen
%	Makros f�r Einheiten, Exponenten
%	Makros f�r Formeln
%	Makros f�r Entwurf




% Selbstdefinierte Trennregeln
% ============================
% Ausnahmen von der automatischen Silbentrennung werden mit dem Befehl
% \hyphenation definiert und gelten f�r das ganze Dokument.
	\hyphenation{Aktu-ali-sie-rung Screen-shots MATLAB}



% Allgemeine Abk�rzungen
% ======================
	\newcommand{\bzw}{bzw.\@\xspace}
	\newcommand{\Bzw}{Bzw.\@\xspace}
	\newcommand{\bzgl}{bzgl.\@\xspace}
	\newcommand{\ca}{ca.\@\xspace}
	\newcommand{\dah}{d.\thinspace{}h.\@\xspace}
	\newcommand{\Dah}{D.\thinspace{}h.\@\xspace}
	\newcommand{\ds}{d.\thinspace{}s.\@\xspace}
	\newcommand{\evtl}{evtl.\@\xspace}
	\newcommand{\ua}{u.\thinspace{}a.\@\xspace}
	\newcommand{\Ua}{U.\thinspace{}a.\@\xspace}
	\newcommand{\uU}{u.\thinspace{}U.\@\xspace}
	\newcommand{\usw}{usw.\@\xspace}
	\newcommand{\etc}{etc.\@\xspace}
	\newcommand{\va}{v.\thinspace{}a.\@\xspace}
	\newcommand{\Vgl}{Vgl.\@\xspace}
	\newcommand{\vgl}{vgl.\@\xspace}
	\newcommand{\zB}{z.\thinspace{}B.\@\xspace}
	\newcommand{\ZB}{Zum Beispiel\xspace}
	\newcommand{\sa}{s.\thinspace{}a.\@\xspace}
	\newcommand{\ia}{i.\thinspace{}a.\@\xspace}
	\newcommand{\bspw}{bspw.\@\xspace}
	\newcommand{\Bspw}{Bspw.\@\xspace}
	\newcommand{\ggf}{ggf.\@\xspace}
	\newcommand{\Ggf}{Ggf.\@\xspace}
	\newcommand{\zT}{z.\thinspace{}T.\@\xspace}
	\newcommand{\iA}{i.\thinspace{}A.\@\xspace}







% Makros f�r Referenzen (Abbildungen, Zitate, ...)
% ================================================

	% Referenzierung auf Abbildungen, Tabellen, etc. (Hyperref-f�hig)
	\newcommand{\figref}[1]{\hyperref[#1]{\figurename\ \ref*{#1}}}
	\newcommand{\tabref}[1]{\hyperref[#1]{\tablename\ \ref*{#1}}}
	\newcommand{\equref}[1]{\hyperref[#1]{Gl.~(\ref*{#1})}}
	\newcommand{\defref}[1]{\hyperref[#1]{Definition~\ref*{#1}}}
	\newcommand{\thrref}[1]{\hyperref[#1]{Satz~\ref*{#1}}}
	\newcommand{\figvref}[1]{\hyperref[#1]{\figurename}\vref{#1}}
	\newcommand{\tabvref}[1]{\hyperref[#1]{\tablename}\vref{#1}}
	\newcommand{\eqvref}[1]{\hyperref[#1]{Gl.~(\ref*{#1}) auf Seite \pageref*{#1}}}
	\newcommand{\pagerefh}[1]{\hyperref[#1]{Seite~\pageref*{#1}}}
	\newcommand{\secref}[1]{\hyperref[#1]{Abschnitt~\ref*{#1}}}
	\newcommand{\charef}[1]{\hyperref[#1]{Kapitel~\ref*{#1}}}
	\newcommand{\appref}[1]{\hyperref[#1]{Anhang~\ref*{#1}}}
	\newcommand{\lstref}[1]{\hyperref[#1]{Listing~\ref*{#1}}}


	% Zitate mit Seitenangabe in Fu�note
%	\newcommand{\citep}[2]{\cite{#1}\footnote{Seite #2}}
%	\newcommand{\citepp}[2]{\cite{#1}\footnote{Seiten #2}}
	\newcommand{\citep}[2]{\cite{#1} (S. #2)}
	\newcommand{\citepp}[2]{\cite{#1} (S. #2)}
	
	
% Makros f�r Abbildungen
% ======================
	% zum Skalieren nach Ersetzen durch psfrag
	\newcommand{\incgraphicsw}[2]{\resizebox{#1}{!}{\includegraphics{#2}}}
	
	% Kleinere Schriften f�r psfrag
	\newcommand{\psfragt}[4]{\psfrag{#1}[#2][#3]{\tiny{#4}}}
	\newcommand{\psfragsm}[4]{\psfrag{#1}[#2][#3]{\small{#4}}}


% Textbausteine
% =============
	% Produktnamen
	\newcommand{\Matlab}{{\itshape Matlab}}
	\newcommand{\Matlabreg}{\textsc{Matlab}\textsuperscript{\tiny \textregistered}}
	\newcommand{\MatSim}{\textsc{Matlab/Simulink}}
	\newcommand{\Simulink}{\textsc{Simulink}}
	\newcommand{\Simulinkreg}{\textsc{Simulink}\textsuperscript{\tiny \textregistered}}


	% H�ufig ben�tigte Textbausteine
	\newcommand{\BSB}{Blockschaltbild}


	\newcommand{\x}{$\times$}


% Makros f�r Einheiten, Exponenten
% ================================

	% Einheit in eckigen Klamnmen, aus Text- UND Math-Modus
	\newcommand{\unitbr}[1] { \ensuremath{\ [ \mathrm{#1} ] }}

	\newcommand{\unit}[1] { \ensuremath{\mathrm{#1}}}
	
	% Wert mit Einheit (mit kleinem Leerzeichen dazwischen), aus Text- UND Math-Modus
	\newcommand{\valunit}[2]{\ensuremath{#1\,\mrm{#2}}}
	\newcommand{\vu}[2]{\ensuremath{#1\,\mrm{#2}}}
	
	% "�C", im Text- oder Mathe-Modus
	\newcommand{\degC}{
		\ifmmode
			^\circ \mrm{C}%
		\else
			\textdegree C%
		\fi}

	\newcommand{\degree}{
		\ifmmode
			^\circ%
		\else
			\textdegree%
		\fi}
	
	% F�r Exponentenschreibweise ( Anwendung: 123\E{3} )
%	\newcommand{\e}[2][]{ \ensuremath{#1 \cdot 10^{#2}} }
	\newcommand{\E}[1]{ \ensuremath{\cdot 10^{#1}} }
	\newcommand{\mE}[2]{ \ensuremath{#1 \cdot 10^{#2}} }
	
	\newcommand{\eexp}[1]{ \mathrm{e}^{#1} }
	\newcommand{\iu}{ \mathrm{j} }
	\newcommand{\wal}{ \mathrm{wal} }
	\newcommand{\WT}[1]{ \mathcal{W} \left\{ #1 \right\} }
	\newcommand{\IWT}[1]{ \mathcal{W}^{-1} \left\{ #1 \right\} }
	\newcommand{\todots}{ ,\,\hdots,\, }
	
	\renewcommand{\Re}[1]{\mrm{Re}\left( #1 \right)}
	\renewcommand{\Im}[1]{\mrm{Im}\left( #1 \right)}
	
	\newcommand{\vecop}{\mathrm{vec}}
	
	\newcommand{\coloneq}{\mathrel{\mathop:}=}

% Makros f�r Formeln
% ==================

	\newcommand{\AP} { \mathrm{AP} }
	\newcommand{\doti} {(i)^\cdot}

	% Definition f�r Vektor und Matizen
	\newcommand{\ve}[1]{\ensuremath{\boldsymbol{\mathrm{#1}}}}
	\newcommand{\ma}[1]{\ensuremath{\boldsymbol{\mathrm{#1}}}}

	\newcommand{\veT}[1]{\ensuremath{\boldsymbol{\mathrm{#1}}^\mrm{T}}}
	\newcommand{\maT}[1]{\ensuremath{\boldsymbol{\mathrm{#1}}^\mrm{T}}}
	

	\newcommand{\diag}[1]{\mathrm{diag}\left( #1 \right)}
	\newcommand{\wbar}[1]{\overline{#1}}
	\newcommand{\what}[1]{\widehat{#1}}
	\newcommand{\chkwbar}[1]{\check{\overline{#1}}}
	\newcommand{\chkwhat}[1]{\check{\widehat{#1}}}
	
	\newcommand{\dyadiccirc}[1]{\leftidx{^*}{\what{\ma{ #1 }}}{}}
	
	\newcommand{\inprod}[2]{\langle #1,\,#2 \rangle}
	
	\newcommand{\ul}[1]{\underline{#1}}

	% gerades "d" (z.B. f�r Integral)
	\newcommand{\ud} { \mathrm{d} }
	
	% normaler Text in Formeln
	\newcommand{\tn}[1] { \textnormal{#1} }
	
	% nicht-kursive Schrift in Formeln
	\newcommand{\mrm}[1] { \mathrm{#1}}
	
	% gerades "T" f�r Transponiert
	\newcommand{\transp}{\mathrm{T}}
	
	% gerades "H" f�r konjugiert-komplex transponiert
	\newcommand{\kktransp}{\mathrm{H}}

	% gerades "rg"
	\newcommand{\rang}[1]{\mathrm{rg}(#1)}

	% F�r geklammerte Ausdr�cke mit Index (Subscript)
	% (einmal mit kursiven Index, einmal mit geradem Index)
	\newcommand{\grpsb}[2] { \left( #1 \right)_{#2} }
	\newcommand{\grprsb}[2] { \left( #1 \right)_{\mathrm{#2}} }

	% Ableitungen und Integrale
		% "normale" Ableitung (mit geraden "d"s)
		\newcommand{\normd}[2] { \frac{\mathrm{d} #1 }{\mathrm{d} #2 } }
		\newcommand{\normdat}[3] { \left. \frac{\mathrm{d} #1 }{\mathrm{d} #2 } \right|_{#3} }
	
		% Materielle Ableitung
		\newcommand{\matd}[2] { \frac{\mathrm{D} #1 }{\mathrm{D} #2 } }
		\newcommand{\matdat}[3] { \left. \frac{\mathrm{D} #1 }{\mathrm{D} #2 } \right|_{#3} }
	
		% Partielle Ableitung
		\newcommand{\partiald}[2] { \frac{\partial #1 }{\partial #2 } }
		\newcommand{\partialdat}[3] { \left. \frac{\partial #1 }{\partial #2 } \right|_{#3} }
		
		% Integral mit Grenzen und vermindertem Abstand
		\newcommand{\integral}[4] { \int_{#1}^{#2} \!\! #3 \mathrm{d} {#4} }
	
	
	% H�ufig auftretende, aufwendig zu setzende Funktionen
	\newcommand{\FT}[1] { \mathcal{F} \left\{ #1 \right\} }
	\newcommand{\FTabs}[1]{\left| \mathcal{F} \left\{ #1 \right\} \right|}
	\newcommand{\IFT}[1] { \mathcal{F}^{-1} \left\{ #1 \right\} }
	\newcommand{\DFT}[1]{\mathrm{DFT} \left\{ #1 \right\}}
	\newcommand{\DFTabs}[1]{\left| \mathrm{DFT} \left\{ #1 \right\} \right|}


% Makros f�r Entwurf
% ==================
	\newcommand{\anm}[1]{\textbf{Anmerkung:} \textit{#1}}
	\newcommand{\ToDo}[1]{\textbf{ToDo:} \textit{#1}}
	\newcommand{\Quelle}{\textbf{Quelle!}}
	\newcommand{\citeM}[1]{(\textbf{Messung #1})}
	\newcommand{\citeS}[1]{(\textbf{Simulation #1})}



	\newcommand{\mlfct}[1]{{\tt #1}}
	\newcommand{\mlvar}[1]{{\tt #1}}	
	
	\newcommand{\textcompstdfont}[1]{{\fontfamily{cmr} \fontseries{m} \fontshape{n} \selectfont #1}}
	
	
% Speziell LZV
	\newcommand{\LA}[1]{\mrm{L}_\mrm{A}^{#1}}
	\newcommand{\LAs}[1]{\mrm{L}_\mrm{A}^{*#1}}
	
	
	
	
\newcommand{\randvar}[1]{\{#1\}_\mrm{r}}
\newcommand{\ndistr}[2]{\mrm{N}( #1,\ #2 )}
\newcommand{\expect}[1]{\mrm{E}\{ #1 \}}
\newcommand{\probability}[1]{\mrm{P}\{ #1 \}}
\newcommand{\randvarB}[1]{\left\{#1\right\}_\mrm{r}}
\newcommand{\ndistrB}[2]{\mrm{N}\left( #1,\ #2 \right)}
\newcommand{\expectB}[1]{\mrm{E}\left\{ #1 \right\}}
