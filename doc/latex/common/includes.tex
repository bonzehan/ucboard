
\newif\ifBeamer
\newif\ifReport

\makeatletter
\@ifclassloaded{tudbeamer}{\Beamertrue}{\Beamerfalse}
\@ifclassloaded{beamer}{\Beamertrue}{}
\makeatother

\ifBeamer
	\Reportfalse
\else
	\Reporttrue
\fi


\usepackage{t1enc}			% evtl. dc-Fonts 
%\usepackage[T1]{fontenc}	% Für Silbentrennung bei Wörten mit Sonderzeichen (z.B. Umlaute)
\usepackage[utf8]{inputenc}
%							% Um Sonderzeichen (ä, ß, é, ...) direkt eingeben zu können
\usepackage[english,ngerman]{babel}
							% Für Sprachenspezifisches
							% ngerman ist schon als globale Option definiert

%\usepackage{helvet}			% Helvetica als Standard-Sans-Schriftart
\usepackage[stable]{footmisc}
\usepackage{booktabs}

\newcommand{\mytoprule}{\toprule[\lightrulewidth]}
\newcommand{\mymidrule}{\midrule}
\newcommand{\mybottomrule}{\bottomrule[\lightrulewidth]}


\usepackage{graphicx}		% zum Einbinden von Postscript
\usepackage{psfrag}			% Beschriftung der Bilder
\usepackage{amsmath}		% Mehr mathematischen Formelsatz
%\usepackage{amssymb}		% Mehr mathematische Symbole
%\usepackage{amsthm}
\usepackage{leftidx}


%\usepackage{float}			% Für Parameter [H] bei Fließobjekten
\usepackage{epsfig}			% Um eps-Bilder einzubinden


\ifReport
	\usepackage{scrhack}		% Um Warnung "float@addtolists detected" zu unterdrücken
	\usepackage{subfig}			% Für Unterabbildungen
	%\captionsetup[subtable]{position=top}

	\usepackage{ltxtable} 		% Vereinigt TabularX und Longtable
	\usepackage{lscape} 		% 
	\usepackage{rotating}		% Zum Drehen von Objekten
	\usepackage{bibgerm}		% Für deutsche Literaturverwaltung
	%\usepackage{wrapfig}		% Für kleine Bilder am Rand
	%\usepackage{floatflt}		% Alternative zu wrapfig
	%\usepackage[hang]{caption}	% Damit mehrzeilige Bildunterschriften gut aussehen
	\usepackage[ngerman]{varioref}		% Für vref
	\usepackage{placeins}		% Für \FloatBarrier
\fi

\usepackage{upgreek}		% Für nicht-kursive kleine griechischen Buchstaben

\usepackage{multirow}		% Für mehrzeilige Felder in Tabellen

\usepackage{textcomp}		% Für Sonderzeichen im normalen Text
							% (offensichtlich in tudreport schon eingebunden)


\usepackage{color}			% Für farbigen Text
\usepackage{xspace}
\usepackage{icomma}			% Damit nach Dezimalkommas kein Abstand eingefügt wird
							% (in math-Umgebungen)

\usepackage{cancel}			% Zum Wegstreichen von Gleichungstermen

\usepackage{array}			% Für Zellentyp "m{}" in tabular-Umgebungen (Vertikal zentriert)

%\usepackage{framed}			% Für Rahmen um Text

\ifReport
	\usepackage{enumitem}	% Für \begin{enumerate}[resume] (Weiterführen der Zählung 
							% über mehrere enumerate-Umgebungen
\fi

\usepackage{expdlist}		% Für den Befehl \listpart, um "`normalen"' Text innerhalb
							% von itemize-Umgebungen einzufügen.

\ifReport
	\usepackage[nohints]{minitoc}	% Für abschnittsweise Inhaltsverzeichnisse
	\usepackage{bibunits}			% Für abschnittsweise Literaturverzeichnisse
	%\usepackage[chapterbib]{chapterbib}
	%\usepackage{multibib}
\fi


\usepackage{listings}		% Um formatierten Quellcode einzubinden
\usepackage{moreverb}		% Für Umgebung "`verbatimtab"' (Verbatim mit Tabs)
\renewcommand{\verbatimtabsize}{4\relax}	% Standardtabweite in "`verbatimtab"' 
											% ist 4 Zeichen

\usepackage{cellspace}		% Für gescheiten Abstand von Formeln zu Tabellen-
							% rändern


\usepackage{etex}
%
% Zeichnen von Blockschaltbildern und ähnlichem
\usepackage{tikz}
\usepackage{pgfplots}
\usepackage{pgf}
% Laden der speziellen Tikz-Pakete:

\usepackage[europeanresistors]{circuitikz}

\usetikzlibrary{matrix,shapes,calc,through,positioning,chains,automata,arrows,decorations.pathmorphing,backgrounds,fit,decorations.pathreplacing,patterns}

%\usepackage{arydshln}

% Das Packet hyperref immer als letztes einbinden!
%\usepackage[ps2pdf, colorlinks=false, pdfborder={0 0 0}]{hyperref}
\ifReport
	\usepackage{hyperref}	% Für Verlinkungen im erzeugten pdf
\else					
	%\usepackage[pagelabels=true]{hyperref}	% Für Verlinkungen im erzeugten pdf
	\providecommand\thispdfpagelabel[1]{}
\fi
							