
\chapter{Kommunikation}


\section{Prinzip}


\begin{itemize}
	\item Die Kommunikation des ucboard ist so aufgebaut sein, dass eine Bedienung über ein einfaches Terminalprogramm möglich ist, um ein einfaches Testen und Debuggen zu ermöglichen.
	\item Es ist eine einfache Interaktion möglich. (\ZB zum Kalibrieren der Sensoren.)
	\item Ein einfaches Parsen der vom ucboard gesendeten Daten soll aber ebenso möglich sein.
\end{itemize}


\begin{itemize}
	\item Textbasiert
		\begin{itemize}
			\item Um bei den Messdaten etwas platzeffizienter zu sein, können diese optional als base64 versendet werden.
		\end{itemize}
	\item Keine \bzw nur bei manchen Befehlen optionale Prüfsummen.
	\item Messdaten und Textnachrichten können vom ucboard ohne Aufforderung versendet werden.
	\item Ansonsten reagiert das ucboard auf Befehle.
\end{itemize}

\begin{itemize}
	\item Die Nachrichten zum ucboard sollen sich möglichst einfach Parsen lassen. Die Nachrichten bestehen aus einem oder mehreren, durch Leerzeichen getrennte Wörter. Der Typ eines Wortes ergibt sich aus dem ersten Zeichen:
		\begin{itemize}
			\item \texttt{A} - \texttt{Z} und \texttt{\_}: String
			\item \verb+~+: Optionsname
			\item \texttt{0} - \texttt{9} und \texttt{-}: Zahl
		\end{itemize}
\end{itemize}



\section{Prinzpielles}

\begin{itemize}
	\item Alle besonderen Startzeichen dürfen auch innerhalb der Nachrichten verwendet werden. Dort haben sie keine besondere Bedeutung.
	\item Besteht eine Nachricht aus mehreren Wörtern, so spielt die Anzahl der Leerzeichen zwischen den Wörtern keine Rolle. (Mindestens eines!)
	\item Nach dem Startzeichen kann, muss aber kein Leerzeichen folgen.
\end{itemize}


\subsubsection{Zu ucboard}
\begin{itemize}
	\item Befehle beginnen mit \texttt{!}
	\item Abfragen beginnen mit \texttt{?}
	\item Antworten auf Fragen des ucboard (im interaktiven Modus) besitzen keinen speziellen Beginn. (Sie dürfen aber auch mit \texttt{!} oder \texttt{?} beginnen.
	\item Das Ende einer Nachricht wird ausschließlich durch \texttt{\textbackslash n} (newline) markiert
	\item Optionen beginnen mit \verb+~+. Wenn der Option ein Wert zugeordnet wird, dann wird dieser nach einem \texttt{=} angeschlossen. Dabei dürfen um das Gleichheitszeichen herum keine Leerzeichen stehen!
\end{itemize}


\subsubsection{Von ucboard}
\begin{itemize}
	\item Direkte Antworten beginnen mit \texttt{:}
	\item Auch jeder Schreibbefehl sollte eine kurze Antwort zur Quittierung senden, \zB \verb+:ok\n+. Es wäre empfehlenswert, den gesetzten Wert zu wiederholen
	\item Fehler bei der Abarbeitung von Befehlen beginnen mit \texttt{:ERR(code)}, wobei \texttt{code} eine positive Ganzzahl als Fehlercode ist. Optional kann nach einem weiteren Doppelpunkt eine Beschreibung des Fehlers folgen: \texttt{:ERR(code):Beschreibung} 
	\item Textausgaben (Display-Funktion) beginnen mit \texttt{'}
	\item Fehlermeldungen sind Textausgaben. Diese beginnen mit \texttt{'ERR(code)}, wobei \texttt{code} eine positive Ganzzahl als Fehlercode ist. Optional kann nach einem weiteren Doppelpunkt eine Beschreibung des Fehlers folgen: \texttt{'ERR(code):Beschreibung} 
	\item Ohne Auf"|forderung versendete Messdaten beginnen mit \texttt{\#} (base64- oder hex-codiert) \bzw \texttt{\#\#} (lesbarer Text).
	\item Wenn eine Nutzerinteraktion notwendig ist, dann sollte das letzte Zeichen vor dem Nachrichtendezeichen ein \texttt{?} sein.
	\item Alle Nachrichten vom ucboard haben ETX (0x03) als Endzeichen. (Dadurch ist es möglich, eine mehrzeilige Ausgabe auf dem Terminalprogramm zu erzeugen.)
		\begin{itemize}
			\item Für die bessere Lesbarkeit ist es empfehlenswert, alle Textnachrichten mit einem Zeilenumbruch (vor ETX) abzuschließen. Lediglich bei hex- oder base64-basierten Messdatennachrichten wird dies nicht gemacht, da diese gerade für eine sparsame Kommunikation gedacht sind.
		\end{itemize}
\end{itemize}



\section{Hardware}

Das ucboard verfügt über zwei Schnittstellen für die Kommunikation mit dem PC. Zum einen kann über einen USB-Anschluss eine serielle Kommunikation aufgebaut werden und zum anderen kann eine RS232-Schnittstelle verwendet werden. Die RS232-Schnittstelle wird dabei nicht als Standard-D-SUB-Stecker (9-polig) angeboten, sondern als Wannensteckeranschluss. Dieser ist so belegt, dass eine direkte Verbindung zu den Anschlüssen des Onboard-PC über ein Flachbandkabel erfolgen kann.

Der dritte, "`reine"' UART-Anschluss ist für den Anschluss möglicher Erweiterungen und nicht die Kommunikation mit dem PC vorgesehen.\footnote{Man könnte auch noch darüber nachdenken, die dritte UART auch für die Kommunikation mit einem PC zu verwenden, da an dieser einfach ein Bluetooth-Adapter angeschlossen werden könnte. Damit wäre es möglich, auch ohne den Onboard-PC das Fahrzeug ferngesteuert zu betreiben.}


Die Parameter der Schnittstellen

\begin{table}%
	\centering
	\caption{Einstellungen serielle Schnittstellen}
	\label{tab:Comm:UARTParam}
	\begin{tabular}{lcr}
			\mytoprule
			 & USB & RS232 \\
			\mymidrule
			Baudrate & 57\,600 & 57\,600 \\
			Datenbits & 8 & 8 \\
			Stopbits & 1 & 1 \\
			Parity & keine & keine \\
			Hardware-Flowcontrol & Nein & Nein \\
			\mybottomrule
	\end{tabular}\\
	\color[rgb]{1,0,0}{ToDo: Testen, welche Baudrate jeweils maximal möglich ist!} \textcolor[rgb]{0.75,0.75,0.75}{\footnotesize{Man könnte dies auch über einen Befehl einstellbar machen, und ein Rücksetzen darüber erreichen, dass beim Starten des ucboards \zB der Taster A betätigt wird. Aber das erscheint mir für diesen Zweck unnötig.}}
\end{table}


\section{Befehle}

\subsection{Übersicht}

\begin{table}[htbp]%
	\centering
	\caption{Übersicht über Hauptbefehle ucboard}
	\label{tab:Comm:Cmds}
	\begin{tabular}{ll}
		\mytoprule
		\verb|RESET| & Neustart ucboard \\
		\verb|ID|	& Abfragen der Fahrzeug-ID \\
		\verb|SID| & Setzen und Abfragen der Session-ID \\
		\verb|TICS| & Abfrage ucboard-Zeit (Millisekunden nach (Neu)start) \\
		\verb|STEER| & Setzen und Abfragen des (Soll-)Lenkwinkels \\
		\verb|DRV| & Setzen und Abfragen der Fahrgeschwindigkeit \\
		\verb|DAQ| & Datenerfassung \\
		\textcolor[rgb]{0.75,0.75,0.75}{\texttt{IMU}} & \textcolor[rgb]{0.75,0.75,0.75}{Zum Parametrieren der IMU} \\
		\textcolor[rgb]{0.75,0.75,0.75}{\texttt{US}} & \textcolor[rgb]{0.75,0.75,0.75}{Zum Parametrieren der US-Sensoren}\\
		\mybottomrule
	\end{tabular}
\end{table}


\subsection{Beschreibung}

\subsubsection{RESET}

Führt einen Neustart des ucboards durch.


\begin{verbatim}
	!RESET NOW
\end{verbatim}


\subsubsection{ID}

Zur Abfrage der Fahrzeug-ID, also der Nummer, die über die Dipschalter auf der Platine eingestellt wird.


\begin{verbatim}
	?ID
	:4
\end{verbatim}


\subsubsection{SID}

Zum Abfragen und Setzen der Session-ID. Dies ist einfach eine Zahl, die nach einem Neustart des ucboards auf 0 gesetzt wird, und der beliebige \verb|int32|-Werte gegeben werden können.


\begin{verbatim}
	?SID
	:0
\end{verbatim}

\begin{verbatim}
	!SID 25363
	:25363
\end{verbatim}

\begin{verbatim}
	?SID
	:25363
\end{verbatim}




\subsubsection{STEER}

Zum Setzen und Abfragen des Soll-Lenkwinkels (Servo-Vorgabe). Der Wert muss eine Ganzzahl zwischen $-1\,000$ und $1\,000$ sein.

Setzen:
\begin{verbatim}
	!STEER -200
	:-200
\end{verbatim}

\textcolor[rgb]{0.75,0.75,0.75}{Optionale Wiederholung des Arguments, um Fehlübertragungen zu detektieren:}
\begin{verbatim}
	!STEER -200 -200
	:-200
\end{verbatim}

\begin{verbatim}
	!STEER -200 -201
	:ERR(1001):Msg corrupted
\end{verbatim}

Abfragen:
\begin{verbatim}
	?STEER
	:-200
\end{verbatim}



\subsubsection{DRV}

Zum Setzen und Abfragen der Fahrgeschwindigkeit. Der Wert muss eine Ganzzahl zwischen $-1\,000$ und 1\,000 oder \verb|OFF| sein.

Setzen:
\begin{verbatim}
	!DRV 300
	:300
\end{verbatim}


\textcolor[rgb]{0.75,0.75,0.75}{Optionale Wiederholung des Arguments, um Fehlübertragungen zu detektieren:}
\begin{verbatim}
	!DRV 300 300
	:300
\end{verbatim}


Ausschalten des Fahrtenreglers:
\begin{verbatim}
	!DRV OFF
	:OFF
\end{verbatim}
Der Unterschied zwischen den Werten 0 und \verb|OFF| liegt darin, dass bei \verb|OFF| der Fahrtenregler vom Fahrakku getrennt wird, also tatsächlich ausgeschaltet ist. Der Fahrtenregler wird bei Übermittlung des nächsten Sollwertes ungleich \verb|OFF| wieder eingeschaltet, jedoch dauert dies einen Moment. (Das Einschalten wird auch mit einem Piepston des Fahrtenreglers begleitet.)


Abfragen:
\begin{verbatim}
	?DRV
	:300
\end{verbatim}


\textcolor[rgb]{0.75,0.75,0.75}{DRV kann auch ohne Parameter aufgerufen werden. Dann dient er nur dazu, die Zeitzählung der Totmannschaltung (\bzw Tot-PC-Schaltung) neu zu starten.
}
\begin{verbatim}
	!DRV
	:ok
\end{verbatim}




\paragraph{DMS}

\textcolor[rgb]{0.75,0.75,0.75}{
"`Totmannschaltung"' (Dead-man switch)\\
\\
Dieser Parameter gibt die Zeit in Millisekunden an, innerhalb derer eine neue DRV-Nachricht erhalten sein muss. Wird diese Zeit ohne DRV-Nachricht überschritten, so wird der Motor gestoppt, \dah der Sollwert auf 0 (aber nicht \texttt{OFF} gesetzt.) (Bei der nächsten \texttt{DRV}-Nachricht wird der Motor dann wieder angesteuert.)
}
\begin{verbatim}
	!DRV ~DMS=1000
	:ok
\end{verbatim}
\textcolor[rgb]{0.75,0.75,0.75}{
Zum Abschalten der Sicherheitsschaltung dient der Wert \texttt{OFF}.}



\subsubsection{DAQ}

\begin{table}[htbp]%
	\centering
	\caption{Signale}
	\label{tab:Comm:DAQ:Signals}
	\begin{tabular}{lllll}
		\mytoprule
		Signal & Beschreibung & Einheit & Datentyp & Länge \\
		\mymidrule
		\verb|AX| & Beschleunigung $x$-Richtung &  & \verb|int16_t| & 2 \\
		\verb|AY| & Beschleunigung $y$-Richtung &  & \verb|int16_t| & 2 \\
		\verb|AZ| & Beschleunigung $z$-Richtung &  & \verb|int16_t| & 2 \\
		\verb|GX| & Drehrate um $x$-Achse & & \verb|int16_t| & 2 \\
		\verb|GY| & Drehrate um $y$-Achse & & \verb|int16_t| & 2 \\
		\verb|GZ| & Drehrate um $z$-Achse & & \verb|int16_t| & 2 \\
		\verb|USF| & Abstand vorne & mm & \verb|uint16_t| & \\
		\verb|USL| & Abstand links & mm & \verb|uint16_t| & \\
		\verb|USR| & Abstand rechts & mm & \verb|uint16_t| & \\
		\verb|VSBAT| & Spannung Systemakku & mV & \verb|uint16_t| & 2\\
		\verb|VDBAT| & Spannung Fahrakku & mV & \verb|uint16_t| & 2\\
		\mybottomrule
	\end{tabular}
\end{table}



\begin{table}[htbp]%
	\centering
	\caption{}
	\label{tab:Comm:DAQ:SpecialValues}
	\subfloat[Binärcodierung]{%
		\begin{tabular}{lcccccc}
			\mytoprule
			Datentyp & \texttt{uint32\_t} & \texttt{int32\_t} & \texttt{uint16\_t} & \texttt{int16\_t} & \texttt{uint8\_t} & \texttt{int8\_t} \\
			\mymidrule
			Keine Daten vorhanden & \texttt{0xFFFFFFFF} & \texttt{0x7FFFFFFF} & \texttt{0xFFFF} & \texttt{0x7FFF} & \texttt{0xFF} & \texttt{0x7F} \\
			Messfehler & \texttt{0xFFFFFFFE} & \texttt{0x7FFFFFFE} & \texttt{0xFFFE} & \texttt{0x7FFE} & \texttt{0xFE} & \texttt{0x7E} \\
			Sensorfehler & \texttt{0xFFFFFFFD} & \texttt{0x7FFFFFFD} & \texttt{0xFFFD} & \texttt{0x7FFD} & \texttt{0xFD} & \texttt{0x7D} \\
			Wert zu groß oder zu klein  & \texttt{0xFFFFFFFC} & \texttt{0x7FFFFFFC} & \texttt{0xFFFC} & \texttt{0x7FFC} & \texttt{0xFC} & \texttt{0x7C} \\
			Wert zu groß  & \texttt{0xFFFFFFFC} & \texttt{0x7FFFFFFC} & \texttt{0xFFFC} & \texttt{0x7FFC} & \texttt{0xFC} & \texttt{0x7C} \\
			Wert zu klein & \texttt{0xFFFFFFFB} & \texttt{0x7FFFFFFB} & \texttt{0xFFFB} & \texttt{0x7FFB} & \texttt{0xFB} & \texttt{0x7B} \\
			\mybottomrule
		\end{tabular}}\\
	\subfloat[Textausgabe]{%
		\begin{tabular}{lc}
			\mytoprule
			 & \\
			\mymidrule
			Keine Daten vorhanden & \texttt{[-{}-{}-]}  \\
			Messfehler & \texttt{[mfault]}  \\
			Sensorfehler & \texttt{[fault]} \\
			Wert zu groß oder zu klein & \texttt{[range]} \\
			Wert zu groß  & \texttt{[over]}  \\
			Wert zu klein & \texttt{[under]} \\
			\mybottomrule
		\end{tabular}}
\end{table}


\textcolor[rgb]{1,0,0}{ToDo: Befehl anpassen (CHS -> CH)}
\begin{verbatim}
	?DAQ CH
	:[...]
\end{verbatim}

\textcolor[rgb]{1,0,0}{ToDo:}
\begin{verbatim}
	?DAQ CH AX
	:[...]
\end{verbatim}



Einzelabfrage von Werten:
\begin{verbatim}
	!DAQ GET USF
	:453 20
\end{verbatim}


\begin{verbatim}
	!DAQ GET ~AGE USF
	:453 20
\end{verbatim}

Erster Rückgabewert ist Wert (Einheit siehe Tabelle), zweiter Wert ist Alter in Tics

\begin{verbatim}
	!DAQ GET ~AGE USF USL USR
	:453 20 1004 30 323 15
\end{verbatim}



\begin{verbatim}
	!DAQ GET ~TICS USF
	:453 14533
\end{verbatim}


\paragraph{Automatische Messgruppen}

Es stehen zehn parametrierbare Messgruppen zur Verfügung. In diesen können verschiedene Sensorwerte zusammengefasst werden.

Paket 1 enthält die Werte des Ultraschalls. Es wird gesendet, wenn alle Ultraschallwerte vorliegen, wobei maximal \valunit{10}{ms} nach dem Erfassen des ersten Wertes gewartet wird. Die Daten werden base64-codiert und mit einer CRC16-Prüfsumme verschickt.
\begin{verbatim}
	!DAQ MKPKG 1 ~ALL=100 ~B64 ~CRC _TIC USF USL USR 
\end{verbatim}

Paket 2 enthält die Spannungen der beiden Akkus, wobei immer eine Nachricht verschickt wird, sobald eine neue Spannung gemessen ist.
\begin{verbatim}
	!DAQ MKPKG 2 ~ANY VSBAT VDBAT 
\end{verbatim}

Paket 3 enthält die Beschleunigungswerte in der Ebene und die Gierrate. Die Daten werden alle \valunit{10}{ms} verschickt. Dabei werden alle innerhalb der \valunit{10}{ms} erfassten Daten gemittelt.
\begin{verbatim}
	!DAQ MKPKG 3 ~TS=10 ~AVG ~B64 ~CRC16 AX AY GZ 
\end{verbatim}


Optionen:
\begin{itemize}
	\item \verb+~ALL[=maxwait]+: Sendet, wenn alle Daten vorhanden sind. Wenn maxwait gegeben ist, dann wird nach dem ersten neuen Wert maximal diese Zeit in Millisekunden gewartet, bis die Daten verschickt werden.\footnote{Dies ist \zB für die US-Sensoren gedacht, deren Daten meist an nacheinander liegenden Zeitschritten ankommen.}
	\item \verb+~ANY+: Sendet, wenn ein neues Datum vorhanden ist.
	\item \verb+~ENC=B64|HEX|ASCII+: Nachricht wird base64-codiert (ohne Padding), Hex-codiert oder Ascii-codiert. ASCII meint hierbei "`lesbaren Text"'. Standardwert ist ASCII.
	\item \verb+~CRC+: CRC16-Prüfsumme
	\item \verb+~TS=Ts+: \textcolor[rgb]{0.75,0.75,0.75}{Abtastzeit in Millisekunden. (Muss ein ganzes Vielfaches aller Kanäle des Paketes sein.)}
	\item \verb+~AVG[=Ta]+: \textcolor[rgb]{0.75,0.75,0.75}{Mittelung über Ta. Ta darf dabei maximal TS sein. Wenn kein Ta angegeben ist, dann wird Ta = Ts gesetzt.}
\end{itemize}


\begin{table}[htbp]%
	\centering
	\caption{}
	\label{tab:Comm:DAQ:SpecialChannels}
	\begin{tabular}{lp{10cm}lll}
		\mytoprule
		Signal & Beschreibung & & Datentyp & Länge \\
		\mymidrule
		\verb|_TIC| & ucboard-Zeit (tics) der Erfassung des ersten Datums & ms & \verb|uint32_t| & 4 \\
		\verb|_TIC16| & die niederwertigen \valunit{16}{bits} von \verb|_TIC| & ms & \verb|uint16_t| & 2 \\
		\verb|_TIC8| & die niederwertigen \valunit{8}{bits} von \verb|_TIC| & ms & \verb|uint8_t| & 1 \\
		\verb|_DTICS| & Delta der ucboard-Zeit zwischen ersten und letztem Datum & ms & \verb|uint32_t| & 4 \\
		\verb|_DTICS16| & Delta der ucboard-Zeit zwischen ersten und letztem Datum, Maximalwert 65\,535 & ms & \verb|uint16_t| & 2 \\
		\verb|_DTICS8| & Delta der ucboard-Zeit zwischen ersten und letztem Datum, Maximalwert 255 (sättigend) & ms & \verb|uint8_t| & 1 \\
		\mybottomrule
	\end{tabular}
	\textcolor[rgb]{1,0,0}{ToDo: Implementierung DTICS16/8 an Tabelle anpassen.}
\end{table}


\textcolor[rgb]{0.75,0.75,0.75}{ToDo:}
\begin{verbatim}
	?DAQ PKG 
	:[...]
\end{verbatim}


\textcolor[rgb]{0.75,0.75,0.75}{ToDo:}
\begin{verbatim}
	?DAQ PKG 1
	:[...]
\end{verbatim}


\textcolor[rgb]{0.75,0.75,0.75}{ToDo:}
\begin{verbatim}
	!DAQ PKG 1 
	:ok
\end{verbatim}


\textcolor[rgb]{0.75,0.75,0.75}{ToDo:}
\begin{verbatim}
	!DAQ DEACTPKG 1
	:ok
\end{verbatim}



\begin{verbatim}
	!DAQ START
	:ok
\end{verbatim}


\begin{verbatim}
	!DAQ STOP
	:ok
\end{verbatim}



\subsubsection{IMU CAL}






