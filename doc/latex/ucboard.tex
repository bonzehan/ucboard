\documentclass[11pt,twoside,colorback,accentcolor=tud2c,nopartpage,bigchapter,fleqn,ngerman]{tudreport}
% Für Skripte 11pt-Schrift verwenden (Standard ist 9.5)


% Für Entwurf auf Rechnern ohne installierte TUDdesign-Pakete
% scrreprt kennt keine Schriftgröße 9.5pt
%\documentclass[10pt,twoside,fleqn,ngerman]{scrreprt}


% =================================================================================
% Falls scrreprt anstelle von tudreport gewählt, muss das über diesen Schalter
% mitgeteilt werden!
% =================================================================================

\newif\ifStdClassDraft
\StdClassDraftfalse	% tudreport
%\StdClassDrafttrue		% scrreprt

% =================================================================================



% =================================================================================
% Definitionen aus tudreport-Vorlage
% =================================================================================
\newif\ifTUDmargin
%\TUDmargintrue		% Sehr breiter rechter Rand
\TUDmarginfalse		% Schmaler (normaler) rechter Rand

\ifTUDmargin	% ggf. breiten Rand setzen
  \geometry{marginparsep=4.2mm,marginparwidth=28.64mm}
\fi

\newlength{\longtablewidth}
\setlength{\longtablewidth}{0.7\linewidth}
\addtolength{\longtablewidth}{-\marginparsep}
\addtolength{\longtablewidth}{-\marginparwidth}
% =================================================================================



% =================================================================================
% Hauptdefintionen sind aus Platzgründen ausgelagert
% =================================================================================

% Einbinden wichtiger und weniger wichtiger Pakete

\newif\ifBeamer
\newif\ifReport

\makeatletter
\@ifclassloaded{tudbeamer}{\Beamertrue}{\Beamerfalse}
\@ifclassloaded{beamer}{\Beamertrue}{}
\makeatother

\ifBeamer
	\Reportfalse
\else
	\Reporttrue
\fi


\usepackage{t1enc}			% evtl. dc-Fonts 
%\usepackage[T1]{fontenc}	% Für Silbentrennung bei Wörten mit Sonderzeichen (z.B. Umlaute)
\usepackage[utf8]{inputenc}
%							% Um Sonderzeichen (ä, ß, é, ...) direkt eingeben zu können
\usepackage[english,ngerman]{babel}
							% Für Sprachenspezifisches
							% ngerman ist schon als globale Option definiert

%\usepackage{helvet}			% Helvetica als Standard-Sans-Schriftart
\usepackage[stable]{footmisc}
\usepackage{booktabs}

\newcommand{\mytoprule}{\toprule[\lightrulewidth]}
\newcommand{\mymidrule}{\midrule}
\newcommand{\mybottomrule}{\bottomrule[\lightrulewidth]}


\usepackage{graphicx}		% zum Einbinden von Postscript
\usepackage{psfrag}			% Beschriftung der Bilder
\usepackage{amsmath}		% Mehr mathematischen Formelsatz
%\usepackage{amssymb}		% Mehr mathematische Symbole
%\usepackage{amsthm}
\usepackage{leftidx}


%\usepackage{float}			% Für Parameter [H] bei Fließobjekten
\usepackage{epsfig}			% Um eps-Bilder einzubinden


\ifReport
	\usepackage{scrhack}		% Um Warnung "float@addtolists detected" zu unterdrücken
	\usepackage{subfig}			% Für Unterabbildungen
	%\captionsetup[subtable]{position=top}

	\usepackage{ltxtable} 		% Vereinigt TabularX und Longtable
	\usepackage{lscape} 		% 
	\usepackage{rotating}		% Zum Drehen von Objekten
	\usepackage{bibgerm}		% Für deutsche Literaturverwaltung
	%\usepackage{wrapfig}		% Für kleine Bilder am Rand
	%\usepackage{floatflt}		% Alternative zu wrapfig
	%\usepackage[hang]{caption}	% Damit mehrzeilige Bildunterschriften gut aussehen
	\usepackage[ngerman]{varioref}		% Für vref
	\usepackage{placeins}		% Für \FloatBarrier
\fi

\usepackage{upgreek}		% Für nicht-kursive kleine griechischen Buchstaben

\usepackage{multirow}		% Für mehrzeilige Felder in Tabellen

\usepackage{textcomp}		% Für Sonderzeichen im normalen Text
							% (offensichtlich in tudreport schon eingebunden)


\usepackage{color}			% Für farbigen Text
\usepackage{xspace}
\usepackage{icomma}			% Damit nach Dezimalkommas kein Abstand eingefügt wird
							% (in math-Umgebungen)

\usepackage{cancel}			% Zum Wegstreichen von Gleichungstermen

\usepackage{array}			% Für Zellentyp "m{}" in tabular-Umgebungen (Vertikal zentriert)

%\usepackage{framed}			% Für Rahmen um Text

\ifReport
	\usepackage{enumitem}	% Für \begin{enumerate}[resume] (Weiterführen der Zählung 
							% über mehrere enumerate-Umgebungen
\fi

\usepackage{expdlist}		% Für den Befehl \listpart, um "`normalen"' Text innerhalb
							% von itemize-Umgebungen einzufügen.

\ifReport
	\usepackage[nohints]{minitoc}	% Für abschnittsweise Inhaltsverzeichnisse
	\usepackage{bibunits}			% Für abschnittsweise Literaturverzeichnisse
	%\usepackage[chapterbib]{chapterbib}
	%\usepackage{multibib}
\fi


\usepackage{listings}		% Um formatierten Quellcode einzubinden
\usepackage{moreverb}		% Für Umgebung "`verbatimtab"' (Verbatim mit Tabs)
\renewcommand{\verbatimtabsize}{4\relax}	% Standardtabweite in "`verbatimtab"' 
											% ist 4 Zeichen

\usepackage{cellspace}		% Für gescheiten Abstand von Formeln zu Tabellen-
							% rändern


\usepackage{etex}
%
% Zeichnen von Blockschaltbildern und ähnlichem
\usepackage{tikz}
\usepackage{pgfplots}
\usepackage{pgf}
% Laden der speziellen Tikz-Pakete:

\usepackage[europeanresistors]{circuitikz}

\usetikzlibrary{matrix,shapes,calc,through,positioning,chains,automata,arrows,decorations.pathmorphing,backgrounds,fit,decorations.pathreplacing,patterns}

%\usepackage{arydshln}

% Das Packet hyperref immer als letztes einbinden!
%\usepackage[ps2pdf, colorlinks=false, pdfborder={0 0 0}]{hyperref}
\ifReport
	\usepackage{hyperref}	% Für Verlinkungen im erzeugten pdf
\else					
	%\usepackage[pagelabels=true]{hyperref}	% Für Verlinkungen im erzeugten pdf
	\providecommand\thispdfpagelabel[1]{}
\fi
							

\ifStdClassDraft
	% Diese Pakete nur extra einbinden, wenn NICHT tudreport als Basis.
	\usepackage{amssymb}
	\usepackage{geometry}
\fi


% Einrichten verschiedener Umgebungen, Listing-Formate, etc.

% =================================================================================
% Informationen (Meta-Daten) f�r pdf
% =================================================================================
\hypersetup{
	pdftitle = {},
	pdfsubject = {},
	pdfkeywords = {},
	pdfcreator = {},
	pdfproducer = {LaTeX with hyperref},
	pdfstartview = {Fit},
	pdfpagelayout = {SinglePage}
}
% =================================================================================


% =================================================================================
% Theorem-Definitionen
% =================================================================================
\ifReport
	\newtheorem{theorem}{Satz}
	\newtheorem{lemma}[theorem]{Lemma}
	\newtheorem{definition}{Definition}
	
	\newtheorem{example}{Beispiel}
\fi
% =================================================================================


%% =================================================================================
%% Umgebungen f�r Aufgabe
%% =================================================================================
%\newcounter{problemnb}[part]      % Neuer Counter bspnummer nummeriert nach Kapitel
%\def\theproblemnb{\thepart.\arabic{problemnb}}
%
%\makeatletter	% Definiert "`@"'-Zeichen um (n�tig f�r \@beginparpenalty)
%\newenvironment{problemexp}%
%	{\refstepcounter{problemnb}%	% Aufgabennummer inkrementieren
%		\vspace{2ex}				% Etwas Platz vor Aufgabe
%		\@beginparpenalty=10000%	% Vor Liste auf keinen Fall umbrechen
%%									% (verhindert, dass "`Aufgabe ..."' alleine
%%									% am Fu� einer Seite steht, und alles andere
%%									% dann auf der n�chsten
%		\par%						% Zeilenumbruch erzwingen
%		\noindent%					% Keinen Einzug
%		\textbf{Aufgabe \theproblemnb\ (Durchf�hrung/Nachbearbeitung):}%
%		\begin{itemize}}%
%	{\end{itemize}%
%		\vspace{3ex}}
%\makeatother	% "`@"'-Zeichen wieder normal
%
%\makeatletter 
%\newenvironment{problemprep}%
%{\refstepcounter{problemnb} \vspace{2ex} \@beginparpenalty=10000 \par \noindent \textbf{Aufgabe \theproblemnb\ (Vorbereitung):}\begin{itemize}}%
%{\end{itemize}\vspace{3ex}}
%\makeatother
%% =================================================================================
%
%
%% =================================================================================
%% Umgebungen f�r L�sung
%% =================================================================================
%\newcommand{\solutionframed}[1]{%
%	\ifSolution %
%		\begin{framed}%
%			\noindent \textbf{L�sung:}\\%
%			#1 %
%		\end{framed}%
%	\fi }
%
%\newcommand{\solutioncomment}[1]{%
%	\ifSolution %
%		\begin{framed}%
%			#1 %
%		\end{framed}%
%	\fi }
%
%
%\makeatletter 
%\newenvironment{solution}[1]%
%{\vspace{2ex} \@beginparpenalty=10000 \par \noindent \textbf{L�sung zu Aufgabe~\ref{#1}:}\begin{itemize}}%
%{\end{itemize}\vspace{3ex}}
%\makeatother
%% =================================================================================


% =================================================================================
% Definitionen f�r pgfplot
% =================================================================================

%\pgfplotsset{compat=1.3}

\pgfplotsset{x tick label style={/pgf/number format/use comma, /pgf/number format/1000 sep=\,},
				y tick label style={/pgf/number format/use comma, /pgf/number format/1000 sep=\,},
				z tick label style={/pgf/number format/use comma, /pgf/number format/1000 sep=\,},
				every axis legend/.append style={nodes={right}},
				every axis/.append style={line width=0ex,solid}}

\pgfplotsset{
	/pgfplots/xlabel near ticks/.style={
		/pgfplots/every axis x label/.style={
			at={(ticklabel cs:0.5)},anchor=near ticklabel
		}
	},
	/pgfplots/ylabel near ticks/.style={
		/pgfplots/every axis y label/.style={
			at={(ticklabel cs:0.5)},rotate=90,anchor=near ticklabel
		}
	}
}

\pgfplotsset{every axis/.append style={line width=.5pt},
				every tick/.append style={line width=0.5pt, black}}
				
%\pgfplotsset{every legend/.append style={/tikz/every row/.append style={row sep=0cm}}}


\pgfplotsset{
		cycle list = {{blue,style=solid}, {red, style=dashed}, {olive,style=solid}, %
						{black,style=dashed}, {purple,style=solid}, {orange,style=dashed}}
}


%\pgfplotsset{
%   mein overlay style/.style={
%    ylabel style={overlay},
%    yticklabel style={overlay}
%  }
%}

% =================================================================================


\tikzstyle{block} = [draw,rectangle,thick,minimum height=2em,minimum width=2em]
\tikzstyle{hblock} = [draw,rectangle,thick,minimum height=3em,minimum width=2em]
\tikzstyle{Hblock} = [draw,rectangle,thick,minimum height=4em,minimum width=2em]
\tikzstyle{sum} = [draw,circle,inner sep=0mm,minimum size=2mm]
\tikzstyle{line} = [semithick]
\tikzstyle{branch} = [circle,inner sep=0pt,minimum size=1mm,fill=black,draw=black]
\tikzstyle{guide} = []
\tikzstyle{connector} = [->,semithick]

\tikzstyle{flowenterexit} = [draw, rounded rectangle, thick, text badly centered, text width=4.5cm]
\tikzstyle{flowsub} = [draw, double, thick, text width = 4cm, minimum height=1.5cm, text centered]
\tikzstyle{flowcond} = [draw, diamond, shape aspect = 2, thick, text badly centered, inner sep=0pt, minimum width=4.5cm, minimum height=1.5cm]
\tikzstyle{flowaction} = [draw, thick, text width = 4cm, minimum height=1.5cm, text centered]
 

\newcommand{\TikZsaturation}{
\begin{tikzpicture}
	% Koordinatensystem
	\draw[->, very thin] (-3mm,0) -- (3mm,0);
	\draw[->, very thin] (0,-3mm) -- (0,3mm);

	% Saturation
	\draw[thin] (-3mm,-1.5mm) -- (-1.5mm,-1.5mm);
	\draw[thin] (-1.5mm,-1.5mm) -- ( 1.5mm, 1.5mm);
	\draw[thin] ( 1.5mm, 1.5mm) -- ( 3mm, 1.5mm);
\end{tikzpicture}
}
%\begin{tikzpicture}
	%% Koordinatensystem
	%\draw[->, very thin] (-3mm,0) -- (3mm,0);
	%\draw[->, very thin] (0,-3mm) -- (0,3mm);
%
	%% Saturation
	%\draw[thin] (-3mm,-1.875mm) -- (-0.75mm,-1.875mm);
	%\draw[thin] (-0.75mm,-1.875mm) -- ( 0.75mm, 1.875mm);
	%\draw[thin] ( 0.75mm, 1.875mm) -- ( 3mm, 1.875mm);
%\end{tikzpicture}
% =================================================================================
% Definitionen f�r Listingsumgebung
% =================================================================================

\lstloadlanguages{Matlab}

\lstdefinestyle{Matlab_colored_smallfont}
{
	language = Matlab,
	tabsize = 4,
	framesep = 3mm,
	frame=tb,
	classoffset = 0,	
	basicstyle = \footnotesize\ttfamily,
	keywordstyle = \bfseries\color[rgb]{0,0,1},
	commentstyle = \itshape\color[rgb]{0.133,0.545,0.133},
	stringstyle = \color[rgb]{0.627,0.126,0.941},
	extendedchars = true,
	breaklines = true,
	prebreak = \textrightarrow,
	postbreak = \textleftarrow,
	%escapeinside = {(*@}{@*)},
	%moredelim = [s][\itshape\color[rgb]{0.5,0.5,0.5}]{[.}{.]},
	numbers = left,
	numberstyle = \tiny,
	stepnumber = 5
}
 
\lstdefinestyle{Matlab_colored}
{
	language = Matlab,
	tabsize = 4,
	framesep = 3mm,
	frame=tb,
	classoffset = 0,	
	basicstyle = \ttfamily,
	keywordstyle = \bfseries\color[rgb]{0,0,1},
	commentstyle = \itshape\color[rgb]{0.133,0.545,0.133},
	stringstyle = \color[rgb]{0.627,0.126,0.941},
	extendedchars = true,
	breaklines = true,
	prebreak = \textrightarrow,
	postbreak = \textleftarrow,
	%escapeinside = {(*@}{@*)},
	%moredelim = [s][\itshape\color[rgb]{0.5,0.5,0.5}]{[.}{.]},
	numbers = left,
	numberstyle = \tiny,
	stepnumber = 5
}                


\lstdefinestyle{C_colored_smallfont}
{
	language=C,
	tabsize = 4,
	framesep = 3mm,
	frame=tb,	
	classoffset = 0,	
	basicstyle = \footnotesize\ttfamily,
	keywordstyle = \bfseries\color[rgb]{0,0,1},
	commentstyle = \itshape\color[rgb]{0.133,0.545,0.133},
	stringstyle = \color[rgb]{0.627,0.126,0.941},
	extendedchars = true,
	breaklines = true,
	prebreak = \textrightarrow,
	postbreak = \textleftarrow,
	%escapeinside = {(*@}{@*)},
	%moredelim = [s][\itshape\color[rgb]{0.5,0.5,0.5}]{[.}{.]},
	numbers = left,
	numberstyle = \tiny,
	stepnumber = 5
}

\lstdefinestyle{C_colored}
{
	language=C,
	tabsize = 4,
	framesep = 3mm,
	frame=tb,
	classoffset = 0,	
	basicstyle = \ttfamily,
	keywordstyle = \bfseries\color[rgb]{0,0,1},
	commentstyle = \itshape\color[rgb]{0.133,0.545,0.133},
	stringstyle = \color[rgb]{0.627,0.126,0.941},
	extendedchars = true,
	breaklines = true,
	prebreak = \textrightarrow,
	postbreak = \textleftarrow,
	%escapeinside = {(*@}{@*)},
	%moredelim = [s][\itshape\color[rgb]{0.5,0.5,0.5}]{[.}{.]},
	numbers = left,
	numberstyle = \tiny,
	stepnumber = 5
}



% Nützliche Befehle und Abkürzungen
% Inhalt
% ======
%	Selbstdefinierte Trennregeln
%	Allgemeine Abk�rzungen
%	Makros f�r Referenzen (Abbildungen, Zitate, ...)
%	Makros f�r Abbildungen
%	Makros f�r Einheiten, Exponenten
%	Makros f�r Formeln
%	Makros f�r Entwurf




% Selbstdefinierte Trennregeln
% ============================
% Ausnahmen von der automatischen Silbentrennung werden mit dem Befehl
% \hyphenation definiert und gelten f�r das ganze Dokument.
	\hyphenation{Aktu-ali-sie-rung Screen-shots MATLAB}



% Allgemeine Abk�rzungen
% ======================
	\newcommand{\bzw}{bzw.\@\xspace}
	\newcommand{\Bzw}{Bzw.\@\xspace}
	\newcommand{\bzgl}{bzgl.\@\xspace}
	\newcommand{\ca}{ca.\@\xspace}
	\newcommand{\dah}{d.\thinspace{}h.\@\xspace}
	\newcommand{\Dah}{D.\thinspace{}h.\@\xspace}
	\newcommand{\ds}{d.\thinspace{}s.\@\xspace}
	\newcommand{\evtl}{evtl.\@\xspace}
	\newcommand{\ua}{u.\thinspace{}a.\@\xspace}
	\newcommand{\Ua}{U.\thinspace{}a.\@\xspace}
	\newcommand{\uU}{u.\thinspace{}U.\@\xspace}
	\newcommand{\usw}{usw.\@\xspace}
	\newcommand{\etc}{etc.\@\xspace}
	\newcommand{\va}{v.\thinspace{}a.\@\xspace}
	\newcommand{\Vgl}{Vgl.\@\xspace}
	\newcommand{\vgl}{vgl.\@\xspace}
	\newcommand{\zB}{z.\thinspace{}B.\@\xspace}
	\newcommand{\ZB}{Zum Beispiel\xspace}
	\newcommand{\sa}{s.\thinspace{}a.\@\xspace}
	\newcommand{\ia}{i.\thinspace{}a.\@\xspace}
	\newcommand{\bspw}{bspw.\@\xspace}
	\newcommand{\Bspw}{Bspw.\@\xspace}
	\newcommand{\ggf}{ggf.\@\xspace}
	\newcommand{\Ggf}{Ggf.\@\xspace}
	\newcommand{\zT}{z.\thinspace{}T.\@\xspace}
	\newcommand{\iA}{i.\thinspace{}A.\@\xspace}







% Makros f�r Referenzen (Abbildungen, Zitate, ...)
% ================================================

	% Referenzierung auf Abbildungen, Tabellen, etc. (Hyperref-f�hig)
	\newcommand{\figref}[1]{\hyperref[#1]{\figurename\ \ref*{#1}}}
	\newcommand{\tabref}[1]{\hyperref[#1]{\tablename\ \ref*{#1}}}
	\newcommand{\equref}[1]{\hyperref[#1]{Gl.~(\ref*{#1})}}
	\newcommand{\defref}[1]{\hyperref[#1]{Definition~\ref*{#1}}}
	\newcommand{\thrref}[1]{\hyperref[#1]{Satz~\ref*{#1}}}
	\newcommand{\figvref}[1]{\hyperref[#1]{\figurename}\vref{#1}}
	\newcommand{\tabvref}[1]{\hyperref[#1]{\tablename}\vref{#1}}
	\newcommand{\eqvref}[1]{\hyperref[#1]{Gl.~(\ref*{#1}) auf Seite \pageref*{#1}}}
	\newcommand{\pagerefh}[1]{\hyperref[#1]{Seite~\pageref*{#1}}}
	\newcommand{\secref}[1]{\hyperref[#1]{Abschnitt~\ref*{#1}}}
	\newcommand{\charef}[1]{\hyperref[#1]{Kapitel~\ref*{#1}}}
	\newcommand{\appref}[1]{\hyperref[#1]{Anhang~\ref*{#1}}}
	\newcommand{\lstref}[1]{\hyperref[#1]{Listing~\ref*{#1}}}


	% Zitate mit Seitenangabe in Fu�note
%	\newcommand{\citep}[2]{\cite{#1}\footnote{Seite #2}}
%	\newcommand{\citepp}[2]{\cite{#1}\footnote{Seiten #2}}
	\newcommand{\citep}[2]{\cite{#1} (S. #2)}
	\newcommand{\citepp}[2]{\cite{#1} (S. #2)}
	
	
% Makros f�r Abbildungen
% ======================
	% zum Skalieren nach Ersetzen durch psfrag
	\newcommand{\incgraphicsw}[2]{\resizebox{#1}{!}{\includegraphics{#2}}}
	
	% Kleinere Schriften f�r psfrag
	\newcommand{\psfragt}[4]{\psfrag{#1}[#2][#3]{\tiny{#4}}}
	\newcommand{\psfragsm}[4]{\psfrag{#1}[#2][#3]{\small{#4}}}


% Textbausteine
% =============
	% Produktnamen
	\newcommand{\Matlab}{{\itshape Matlab}}
	\newcommand{\Matlabreg}{\textsc{Matlab}\textsuperscript{\tiny \textregistered}}
	\newcommand{\MatSim}{\textsc{Matlab/Simulink}}
	\newcommand{\Simulink}{\textsc{Simulink}}
	\newcommand{\Simulinkreg}{\textsc{Simulink}\textsuperscript{\tiny \textregistered}}


	% H�ufig ben�tigte Textbausteine
	\newcommand{\BSB}{Blockschaltbild}


	\newcommand{\x}{$\times$}


% Makros f�r Einheiten, Exponenten
% ================================

	% Einheit in eckigen Klamnmen, aus Text- UND Math-Modus
	\newcommand{\unitbr}[1] { \ensuremath{\ [ \mathrm{#1} ] }}

	\newcommand{\unit}[1] { \ensuremath{\mathrm{#1}}}
	
	% Wert mit Einheit (mit kleinem Leerzeichen dazwischen), aus Text- UND Math-Modus
	\newcommand{\valunit}[2]{\ensuremath{#1\,\mrm{#2}}}
	\newcommand{\vu}[2]{\ensuremath{#1\,\mrm{#2}}}
	
	% "�C", im Text- oder Mathe-Modus
	\newcommand{\degC}{
		\ifmmode
			^\circ \mrm{C}%
		\else
			\textdegree C%
		\fi}

	\newcommand{\degree}{
		\ifmmode
			^\circ%
		\else
			\textdegree%
		\fi}
	
	% F�r Exponentenschreibweise ( Anwendung: 123\E{3} )
%	\newcommand{\e}[2][]{ \ensuremath{#1 \cdot 10^{#2}} }
	\newcommand{\E}[1]{ \ensuremath{\cdot 10^{#1}} }
	\newcommand{\mE}[2]{ \ensuremath{#1 \cdot 10^{#2}} }
	
	\newcommand{\eexp}[1]{ \mathrm{e}^{#1} }
	\newcommand{\iu}{ \mathrm{j} }
	\newcommand{\wal}{ \mathrm{wal} }
	\newcommand{\WT}[1]{ \mathcal{W} \left\{ #1 \right\} }
	\newcommand{\IWT}[1]{ \mathcal{W}^{-1} \left\{ #1 \right\} }
	\newcommand{\todots}{ ,\,\hdots,\, }
	
	\renewcommand{\Re}[1]{\mrm{Re}\left( #1 \right)}
	\renewcommand{\Im}[1]{\mrm{Im}\left( #1 \right)}
	
	\newcommand{\vecop}{\mathrm{vec}}
	
	\newcommand{\coloneq}{\mathrel{\mathop:}=}

% Makros f�r Formeln
% ==================

	\newcommand{\AP} { \mathrm{AP} }
	\newcommand{\doti} {(i)^\cdot}

	% Definition f�r Vektor und Matizen
	\newcommand{\ve}[1]{\ensuremath{\boldsymbol{\mathrm{#1}}}}
	\newcommand{\ma}[1]{\ensuremath{\boldsymbol{\mathrm{#1}}}}

	\newcommand{\veT}[1]{\ensuremath{\boldsymbol{\mathrm{#1}}^\mrm{T}}}
	\newcommand{\maT}[1]{\ensuremath{\boldsymbol{\mathrm{#1}}^\mrm{T}}}
	

	\newcommand{\diag}[1]{\mathrm{diag}\left( #1 \right)}
	\newcommand{\wbar}[1]{\overline{#1}}
	\newcommand{\what}[1]{\widehat{#1}}
	\newcommand{\chkwbar}[1]{\check{\overline{#1}}}
	\newcommand{\chkwhat}[1]{\check{\widehat{#1}}}
	
	\newcommand{\dyadiccirc}[1]{\leftidx{^*}{\what{\ma{ #1 }}}{}}
	
	\newcommand{\inprod}[2]{\langle #1,\,#2 \rangle}
	
	\newcommand{\ul}[1]{\underline{#1}}

	% gerades "d" (z.B. f�r Integral)
	\newcommand{\ud} { \mathrm{d} }
	
	% normaler Text in Formeln
	\newcommand{\tn}[1] { \textnormal{#1} }
	
	% nicht-kursive Schrift in Formeln
	\newcommand{\mrm}[1] { \mathrm{#1}}
	
	% gerades "T" f�r Transponiert
	\newcommand{\transp}{\mathrm{T}}
	
	% gerades "H" f�r konjugiert-komplex transponiert
	\newcommand{\kktransp}{\mathrm{H}}

	% gerades "rg"
	\newcommand{\rang}[1]{\mathrm{rg}(#1)}

	% F�r geklammerte Ausdr�cke mit Index (Subscript)
	% (einmal mit kursiven Index, einmal mit geradem Index)
	\newcommand{\grpsb}[2] { \left( #1 \right)_{#2} }
	\newcommand{\grprsb}[2] { \left( #1 \right)_{\mathrm{#2}} }

	% Ableitungen und Integrale
		% "normale" Ableitung (mit geraden "d"s)
		\newcommand{\normd}[2] { \frac{\mathrm{d} #1 }{\mathrm{d} #2 } }
		\newcommand{\normdat}[3] { \left. \frac{\mathrm{d} #1 }{\mathrm{d} #2 } \right|_{#3} }
	
		% Materielle Ableitung
		\newcommand{\matd}[2] { \frac{\mathrm{D} #1 }{\mathrm{D} #2 } }
		\newcommand{\matdat}[3] { \left. \frac{\mathrm{D} #1 }{\mathrm{D} #2 } \right|_{#3} }
	
		% Partielle Ableitung
		\newcommand{\partiald}[2] { \frac{\partial #1 }{\partial #2 } }
		\newcommand{\partialdat}[3] { \left. \frac{\partial #1 }{\partial #2 } \right|_{#3} }
		
		% Integral mit Grenzen und vermindertem Abstand
		\newcommand{\integral}[4] { \int_{#1}^{#2} \!\! #3 \mathrm{d} {#4} }
	
	
	% H�ufig auftretende, aufwendig zu setzende Funktionen
	\newcommand{\FT}[1] { \mathcal{F} \left\{ #1 \right\} }
	\newcommand{\FTabs}[1]{\left| \mathcal{F} \left\{ #1 \right\} \right|}
	\newcommand{\IFT}[1] { \mathcal{F}^{-1} \left\{ #1 \right\} }
	\newcommand{\DFT}[1]{\mathrm{DFT} \left\{ #1 \right\}}
	\newcommand{\DFTabs}[1]{\left| \mathrm{DFT} \left\{ #1 \right\} \right|}


% Makros f�r Entwurf
% ==================
	\newcommand{\anm}[1]{\textbf{Anmerkung:} \textit{#1}}
	\newcommand{\ToDo}[1]{\textbf{ToDo:} \textit{#1}}
	\newcommand{\Quelle}{\textbf{Quelle!}}
	\newcommand{\citeM}[1]{(\textbf{Messung #1})}
	\newcommand{\citeS}[1]{(\textbf{Simulation #1})}



	\newcommand{\mlfct}[1]{{\tt #1}}
	\newcommand{\mlvar}[1]{{\tt #1}}	
	
	\newcommand{\textcompstdfont}[1]{{\fontfamily{cmr} \fontseries{m} \fontshape{n} \selectfont #1}}
	
	
% Speziell LZV
	\newcommand{\LA}[1]{\mrm{L}_\mrm{A}^{#1}}
	\newcommand{\LAs}[1]{\mrm{L}_\mrm{A}^{*#1}}
	
	
	
	
\newcommand{\randvar}[1]{\{#1\}_\mrm{r}}
\newcommand{\ndistr}[2]{\mrm{N}( #1,\ #2 )}
\newcommand{\expect}[1]{\mrm{E}\{ #1 \}}
\newcommand{\probability}[1]{\mrm{P}\{ #1 \}}
\newcommand{\randvarB}[1]{\left\{#1\right\}_\mrm{r}}
\newcommand{\ndistrB}[2]{\mrm{N}\left( #1,\ #2 \right)}
\newcommand{\expectB}[1]{\mrm{E}\left\{ #1 \right\}}


% =================================================================================



% =================================================================================
% Befehle, die in scrreprt nicht exisiteren werden hier definiert, wenn diese
% Klasse verwendet wird.
% Auch die Seitenränder werden angepasst, so dass es grob wie mit tudreport
% aussieht.
% =================================================================================
\ifStdClassDraft
	\newcommand{\subsubtitle}[1]{}
	\newcommand{\settitlepicture}[1]{}
	\newcommand{\printpicturesize}{}
	\newcommand{\institution}[1]{}
	
	\author{}	% scrreprt erwartet Autor
	
	\geometry{left=20mm, right=15mm, top=15mm, bottom=20mm}
\fi
% =================================================================================



% =================================================================================
% Anpassung Absatzformat
% =================================================================================
% Wenn nur sehr kurze Absätze mit vielen Formeln im Text vorkommen, dann sieht der 
% standardmäßig vorhandene Einzug bei Absatzbeginn etwas unruhig aus. Dann kann
% dieser hier auf Null gesetzt, und dafür der vertikale Abstand zwischen zwei
% Absätzen etwas vergrößert werden.
%
\setlength{\parindent}{0mm}
\setlength{\parskip}{1ex}
% =================================================================================




% =================================================================================
% Texte für den Titel und die Rückseite des Titels vorgeben
% =================================================================================
\title{Beschreibung Fahrzeug und ucboard}
\subtitle{PS Echtzeitsysteme}
\subsubtitle{Rev. 0+ (Entwurf), Stand 26.11.2016}

% \settitlepicture{}	% Bild für Deckblatt
% \printpicturesize
%\setinstitutionlogo[height]{common/rtm_mit_schrift}	% Logo statt \institution-Text
\institution{
	PS Echtzeitsysteme\\FG Echtzeitsysteme\\FG Regelungstechnik und Mechatronik}

% Folgende Einträge werden auf der Rückseite der Titelseite gedruckt:
\uppertitleback{}
\lowertitleback{}
\dedication{}
% =================================================================================


% =================================================================================
% Definition eigener Befehle (Die meisten verwendeten Befehle sind aus
% common/commommacros.tex
% =================================================================================
\newcommand{\eqis}{\mathrel{\widehat{=}}}
\newcommand{\rs}[1]{\textbf{Rücksprache:} #1}
% =================================================================================


\begin{document}
%\selectlanguage{english}
\selectlanguage{ngerman}

\maketitle


%% =================================================================================
%% Seite für Informationen, die nicht mehr auf die Titelseite passen
%% =================================================================================
%\cleardoublepage
%{
	%\setlength{\topskip}{5cm}		% Abstand vom oberen Rand
	%
	%\noindent{\sffamily \huge Notizen}\\[.5cm]
	%\noindent{\sffamily \LARGE Eric Lenz}\\[.5cm]
%
	%%\noindent 
	%
	%\vspace*{\fill}		% Den Rest unten bündig darstellen
	%\vfill
	%
	%\noindent \includegraphics[width=3cm]{./common/IAT_rtm.eps}\\[0.5cm]
	%Technische Universität Darmstadt\\
	%Institut für Automatisierungstechnik und Mechatronik\\
	%Fachgebiet Regelungstechnik und Mechatronik\\
	%Prof. Dr.-Ing. U. Konigorski\\[1cm]
	%%
	%Landgraf-Georg-Straße 4\\
	%64283 Darmstadt\\
	%Telefon 06151/16-3014\\
	%www.rtm.tu-darmstadt.de
%}
%% =================================================================================


% =================================================================================
% Zusammenfassung
% =================================================================================
%\begin{abstract}
%\end{abstract}  
% =================================================================================





% =================================================================================
% Inhaltsverzeichnis
% =================================================================================
\cleardoublepage	% Auf einer leeren rechten Seite beginnen
\phantomsection		% Diese und die nächste Zeile dient dazu, für das Inhalts-
					% verzeichnis einen Eintrag in das pdf-Inhaltsverzeichnis,
					% aber nicht in das normale Verzeichnis zu erzeugen.
\pdfbookmark[0]{\contentsname}{pdf:toc}	
\tableofcontents	% Inhaltsverzeichnis einfügen
\clearpage			% Sonst kommt nichts mehr auf die Seite
% =================================================================================



% =================================================================================
% Hauptteil
% =================================================================================


\part{Anwenden}

\clearpage

\chapter{Fahrzeuge}


\section{Fahrgestelle}

\begin{table}[htb]%
	\centering
	\caption{Technische Daten}
	\label{tab:chassis:data}
	\begin{tabular}{llr}
		\mytoprule
		Radstand & & \valunit{258}{mm} \\
		Spurweite & & \valunit{155}{mm} \\
		Masse & & \valunit{}{kg} \\
		\mybottomrule
	\end{tabular}
\end{table}


\section{Aktorik}

\subsection{Lenkung}

Die Lenkung ist als Achsschenkel"|lenkung ausgeführt und wird über ein normales Modellbauservo angetrieben.

Das Servo stellt (über einen integrierten P-Regler) einen Sollwinkel, welcher sich über eine Mechanik auf einen Lenkwinkel der Räder überträgt.

Die Kennlinie "`Servowinkel - Lenkwinkel"' ist nicht dokumentiert.

Die Ansteuerung des Servos, \dah die Vorgabe eines Soll-Servowinkels, erfolgt über ein PWM-Signal. Dieses hat eine Frequenz von \valunit{50}{Hz} (wobei diese nicht sonderlich wesentlich ist) mit Impulsbreiten $T_\mrm{puls}$ zwischen \valunit{1}{ms} und \valunit{2}{ms} (die wesentlich sind).

Dabei bedeutet eine Impulsbreite von \valunit{1,5}{ms} die Neutralstellung des Servos, welches (näherungsweise) einer Geradeausstellung der Räder entsprechen sollte. Ein Impulsbreite von \valunit{2}{ms} bedeutet eine maximale Auslenkung des Servos gegen den Uhrzeigersinn (bei Blick auf die Servowelle) was hier einer Lenkbewegung nach rechts entspricht. Eine Impulsbreite von \valunit{1}{ms} entspricht damit einer maximalen Lenkbewegung nach links.

In \figref{fig:Steering:Servo} ist das Ansteuersignal für den Servo dargestellt.

\begin{figure}[htb]%
	\centering
	\begin{tikzpicture}
		\begin{axis}[%
				width=12cm,
				height=3cm,
				scale only axis,
				axis x line = center,
				axis y line = center,
				xlabel={$t$ [ms]},
				ylabel={$u$ [--]},
				xmin=-3, xmax=28,
				%xtick={-500, -250, 0, 250, 500},
				ymin=0, ymax=1.3,
				%ytick={-100, -50, 0, 50, 100},
				legend pos = south east]

		\addplot [
				color=black,
				solid,
				forget plot
				]
			coordinates{
				(-10, 0) (0, 0) (0, 1) (1.5, 1) (1.5, 0) (20, 0) (20,1) (21.5,1) (21.5,0) (40,0)
			};
			
		\draw[black,->] (axis cs:19,0.75) -- (axis cs:20,0.75);
		\draw[black,<-] (axis cs:21.5,0.75) -- node[auto]{$T_\mrm{Puls}$} (axis cs:24,0.75);
			

		%\addplot [
				%color=gray!50,
				%solid,
				%forget plot
				%]
			%coordinates{
				%(-10, 0) (0, 0) (0, 1) (2, 1) (2, 0) (20, 0) (20,1) (22,1) (22,0) (40,0)
			%};

		\end{axis}
	\end{tikzpicture}
	
	\caption{Ansteuerung Servo}%
	\label{fig:Steering:Servo}%
\end{figure}


Das ucboard übernimmt die Ansteuerung des Servos. Dabei wird von außen ein Sollwert zwischen $-1\,000$ und $1\,000$ vorgeben. Dieser Bereich wird auf eine Impulsbreite von 1 bis \valunit{2}{ms} umgerechnet. Die Auf"|lösung der gestellten Impulsbreite beträgt dabei \valunit{1}{\upmu s}. (Damit ist die effektive Auf"|lösung der Stellgröße 2.)

Es ist Folgendes zu beachten:
\begin{itemize}
	\item Die Servos können aufgrund der Lenkmechanik nicht ihren vollen Stellbereich ausnutzen. Dies hört man, wenn ein Sollwert von 1\,000 oder $-1\,000$ vorgegeben wird. Es sollte auf Dauer vermieden werden, Servowinkel zu steuern, die nicht erreichbar sind.
	\item Lenken im Stillstand ist wie bei einem normalen Auto schwerer als in der Fahrt. Es kann je nach Achslast sein, dass das Servo einen gewünschten Lenkwinkel im Stillstand nicht erreicht. 
	\item Die Lenkung besitzt ein gewisses Spiel. (Mit der IMU kann aber die Gierrate bestimmt werden, und damit kann bei bekannter Fahrgeschwindigkeit wiederum auf den Lenkwinkel geschlossen werden. Somit könnte der Lenkwinkel unterlagert geregelt werden.)
\end{itemize}




\subsection{Fahrmotor (Fahrtenregler)}
\label{sec:hw:drivectrl}

Die Fahrzeuge verfügen über Gleichstrommotoren, die über einen Tamiya Fahrtenregler angesteuert werden. Dabei ist bei den Chassis 1 bis 5 ein Tamiya TEU-101BK und bei den Chassis 6 und 7 ein Tamiya TEU-104BK verbaut. (Im Wesentlichen unterscheiden diese sich dadurch, dass bei den TEU-104BK ein Batterieschutz implementiert ist, der jedoch hier deaktiviert ist.)

Die Fahrtenregler werden wie die Lenkung durch ein "`Servo-PWM-Signal"' angesteuert. \Dah eine Pulsbreite von \valunit{1,5}{ms} entspricht "`aus"', kürzere Pulsbreiten einer Vorwärtsfahrt und negative Impulsbreiten einer Rückwärtsfahrt \bzw Bremsen.

Die "`Endanschläge"' (sowie die Neutralstellung) sind dabei kalibrierbar. Hier ist es so kalibiriert, dass die Impulsbreiten \valunit{1}{ms} und \valunit{2}{ms} die Grenzwerte und \valunit{1,5}{ms} die Neutralstellung darstellen.

Da der Fahrtenregler intern gewisse Toleranzzonen berücksichtigt sowie in der Rückwärtsfahrt nur die halbe Stellgröße verwendet, ergibt sich die in \figref{fig:DrvCtrl:charact} gezeigte Kennlinie, wobei $\Delta t_\mrm{Puls}$ die Abweichung der Pulsbreite von \valunit{1,5}{ms} ist,
\begin{align*}
	T_\mrm{Puls} = \valunit{1,5}{ms} + \Delta T_\mrm{Puls}\;.
\end{align*}

\begin{figure}[htb]%
	\centering
	\begin{tikzpicture}
		\begin{axis}[%
				width=6cm,
				height=4cm,
				scale only axis,
				axis x line = center,
				axis y line = center,
				xlabel={$\Delta T_\mrm{Puls}$ $[\mrm{\upmu s}]$},
				ylabel={$u$ [\%]},
				xmin=-600, xmax=600,
				xtick={-500, -250, 0, 250, 500},
				ymin=-120, ymax=120,
				ytick={-100, -50, 0, 50, 100},
				legend pos = south east]

		\addplot [
				color=black,
				solid,
				forget plot
				]
			coordinates{
				(-500,100) (-480,100) (-50,0) (0,0) (50,0) (250,-50) (500,-50)
			};

		\end{axis}
	\end{tikzpicture}
	
	\caption{Kennlinie Fahrtenregler}%
	\label{fig:DrvCtrl:charact}%
\end{figure}

Aufgrund des eigentlichen Einsatzzweckes der Fahrtenregler weisen diese folgendes Verhalten auf:
\begin{itemize}
	\item Negative $\Delta T_\mrm{Puls}$, \dah kleinere Pulsbreiten als die Neutralstellung (außerhalb der Toleranzzone), führen immer zur Vorwärtsfahrt.
	\item Wenn von aus einer Vorwärtsbewegung zu positiven $\Delta T_\mrm{Puls}$ gewechselt wird, dann erfolgt eine Bremsung. Diese wird aber nicht über den Stillstand hinaus ausgeführt, \dah der Fahrtenregler stellt sicher, dass keine Rückwärtsbewegung eintritt.
	\item Um ausgehend von einer Vorwärtsfahrt Rückwärts zu fahren, muss der Fahrtenregler einmal mit positivem $\Delta T_\mrm{Puls}$ angesteuert werden (Bremsen), dann muss der Fahrtenregler mit einer Pulsbreite von \valunit{1,5}{ms} angesteuert werden. Damit ist die Rückwärtsfahrt "`freigeschaltet"'. Wenn jetzt wieder ein positives $\Delta T_\mrm{Puls}$ aufgebracht wird, dann erfolgt eine Rückwärtsfahrt. (Die Einzelschritte dieser Sequenz sind mindestens für \valunit{100}{ms} zu halten. Damit hat sich bisher immer eine sichere Umschaltung ergeben.)
\end{itemize}

Das ucboard bietet zur Ansteuerung zwei Möglichkeiten. Zum einen kann der Fahrtenregler "`direkt"' betrieben werden, \dah man dann den Wert für $\Delta T_\mrm{Puls}$ in Mikrosekunden direkt vorgeben. Die zweite Möglichkeit ist die der "`gemanagten"' Ansteuerung. Hierbei wird dem ucboard mitgeteilt, welche Fahrtrichtung gewählt werden soll und welche Stellgröße dabei verwendet werden soll. Dabei meint ein positiver Wert immer die gewählte Bewegungsrichtung, bei Vorwärtsfahrt bedeutet ein negativer Wert eine Bremsung. Dabei bildet bei Vorwärtsfahrt der Wertebereich von 1 bis 1\,000 den Stellgrößenbereich von 0 bis 100\,\% ab, bei Rückwärtsfahrt wird der Wertebereich von 1 bis 500 auf 0 bis $-50\,\%$ abgebildet. Ein Wert von 0 bedeutet Neutralstellung. Es ergeben sich damit die Kennlinien aus \figref{fig:DrvCtrl:charact_fb}.

\begin{figure}[htb]%
	\centering
	\subfloat[Vorwärts/Bremsen\label{fig:DrvCtrl:charact_f}]{%
		\begin{tikzpicture}
			\begin{axis}[%
					width=6cm,
					height=4cm,
					scale only axis,
					axis x line = center,
					axis y line = center,
					xlabel={$u^*$ [--]},
					ylabel={$u$ [\%]},
					xmin=-1100, xmax=1100,
					xtick={-1000, -500, 0, 500, 1000},
					ymin=-120, ymax=120,
					ytick={-100, -50, 0, 50, 100},
					legend pos = south east]

			\addplot [
					color=black,
					solid,
					forget plot
					]
				coordinates{
					(-500,-50) (-1,0) (0,0) (1,0) (1000,100)
				};

			\end{axis}
		\end{tikzpicture}
	}
	\hspace{2cm}
	\subfloat[Rückwärts\label{fig:DrvCtrl:charact_b}]{%
		\begin{tikzpicture}
			\begin{axis}[%
					width=6cm,
					height=4cm,
					scale only axis,
					axis x line = center,
					axis y line = center,
					xlabel={$u^*$ [--]},
					ylabel={$u$ [\%]},
					xmin=-1100, xmax=1100,
					xtick={-1000, -500, 0, 500, 1000},
					ymin=-120, ymax=120,
					ytick={-100, -50, 0, 50, 100},
					legend pos = south east]

			\addplot [
					color=black,
					solid,
					forget plot
					]
				coordinates{
					(0,0) (1,0) (500,-50)
				};

			\end{axis}
		\end{tikzpicture}
	}
	\caption{Kennlinien zur "`gemanagten"' Ansteuerung über ucboard}%
	\label{fig:DrvCtrl:charact_fb}%
\end{figure}

Die für die Umschaltung von Vorwärts- auf Rückwärtsfahrt notwendige Bestimmung des Ist-Zustandes des Fahrtenreglers erfolgt dabei über einen Zustandsautomaten, der die Zustandswechsel des Fahrtenreglers nachbildet. Für den normalen Betrieb sollte dies sicher funktionieren. Fehler können nur dann auf"|treten, wenn
\begin{itemize}
	\item Sehr schnell zwischen Vor- und Rückwärtsfahrt gewechselt wird (weniger als \valunit{200}{ms} zwischen aufeinanderfolgenden Wechsel).
	\item Im Direktmodus Werte gestellt werden, die an den Rändern des Toleranzbereichs für die Neutralstellung liegen.
\end{itemize}
Erfolgt eine Vorwärtsfahrt (länger als \valunit{100}{ms}) stimmen die Zustände wieder überein.




Die Fahrtenregler sind hier so angeschlossen, dass \emph{keine} der \valunit{5}{V}-Spannungsversorgungsleitungen angeschlossen ist, sondern die Spannung direkt aus dem Fahrakku nimmt. Die Verbindung zum Fahrakku kann über das ucboard und den drvbatswitch geschaltet werden.



Die notwendige Massenverbindung als Bezug für das PWM-Signal wird über die Verbindung des ucboards mit dem drvbatswitch hergestellt. Dieses muss daher immer angeschlossen sein. (Ansonsten muss (nur!) die Masseleitung des nicht angeschlossenen Spannungsversorgungskabel des Fahrtenreglers an einen Massepin des ucboards angeschlossen werden.) 



\paragraph{Problemlösung}

\begin{itemize}
	\item Schalter des Fahrtenreglers am Chassis auf "`ON"'?
\end{itemize}



\section{Sensorik}

\subsection{Ultraschall}

Es sind drei Ultraschallsensoren des Typs SRF08 verbaut, jeweils einer nach vorne, nach links und nach rechts.

Es wird die Zeit $T_\mrm{echo}$ vom Aussenden des Ultraschallimpulses bis zum Empfang des ersten Echos gemessen.\footnote{Genau genommen werden auch noch möglicherweise auf"|tretende weitere Echos gemessen. Die dazugehörigen Zeiten stellt der Sensor in weiteren Registern zur Verfügung. Aktuell werden diese jedoch nicht ausgelesen. Möglicherweise könnte man mit einer Auswertung dieser Daten Fehlmessungen reduzieren.}

Der gemessene Abstand $d$ entspricht der halben Signallaufstrecke $c \cdot T_\mrm{echo}$, wobei für die Schallgeschwindigkeit $c = \valunit{343,2}{m/s}$ angenommmen wird. Es ergibt sich damit
\begin{align*}
	d = \frac{1}{2} \cdot c \cdot T_\mrm{echo} = \valunit{171,6}{\frac{m}{s}} \cdot T_\mrm{echo}\;.
\end{align*}


Der Sensor gibt die Zeiten $T_\mrm{echo}$ in Mikrosekunden zurück, wobei die Auf"|lösung \valunit{4}{\upmu s} ist. Damit ergibt sich eine Distanzauflösung von \ca \valunit{0,7}{mm}. Die Messwerte werden vom ucboard in \valunit{0,1}{mm} zurückgegeben.


\paragraph{Optionen: Messbereich und Verstärkung}

Siehe Datenblatt des Sensors.


\subsection{Hall-Sensor}

Mit dem Hall-Sensor (Typ HAL 503) kann die Drehzahl des hinteren linken Rades gemessen werden. Dazu sind auf der Felge des Rades über den Umfang acht Magnete verteilt, die von der Polarität her immer wechselweise angeordnet sind.

Der genannte Sensor besitzt zwei stabile Zustände. Wird ein positiver magnetischer Pol in die Nähe gebracht, so wechselt er in den Zustand 1, bei einem negativen magnetischen Pol in den Zustand 0. Liegt kein Feld vor, dann wird der alte Zustand gehalten. 

Somit kann durch Messen der Zeit zwischen zwei Zustandswechseln die Zeitdauer für eine achtel Umdrehung bestimmt werden. 

Auf dem ucboard wird die Zeit zwischen den Impulsen auf eine Mikrosekunde genau bestimmt. Die Ausgabe der Messwerte erfolgt dann in \valunit{0,1}{ms}. Daneben wird auch die Summe dieser Zeiten über acht Zustandswechsel, also einer Radumdrehung, bestimmt und in einer Genauigkeit von \valunit{1}{ms} ausgegeben. Zuletzt erfolgt auch die Ausgabe der Anzahl an gezählten Zustandswechseln in Modulo 256, wenn extern selber mitgezählt werden soll. (Letzteres erlaubt auch die Überprüfung, ob ggf. Messdaten verloren gegangen sind.)

Im Gegensatz zu einem "`klassischen"' Encoder kann die Drehrichtung nicht festgestellt werden!

\begin{itemize}
	\item Der Abstand zwischen Rad und Aufbau darf nicht zu groß sein. Ist das Fahrzeug beispielsweise aufgebockt, so werden in der Regler keine Impulse mehr gezählt.\footnote{Vom Hersteller gibt es noch einen Hall-Sensor aus der gleichen Baureihe (HAL 502), der jedoch etwas empfindlicher (die notwendige Flussdichte beträgt \ca ein Drittel im Vergleich zum HAL 503) ist. Sollten häufig Probleme mit übersprungenen Impulsen auf"|treten, wäre dies eine mögliche Lösung.}
\end{itemize}


\subsection{IMU}

Auf der hinteren rechten Ecke des ucboards befindet sich ein Beschleunigungs- und Drehratensensor (IMU -- Inertial Measurement Unit) des Typs MPU-9250A.

Die $x$-Achse zeigt nach vorne, die $y$-Achse nach links und die $z$-Achse nach oben.

Der Sensor gibt die Messwerte als \texttt{int16\_t}-Werte aus. Die Umrechnung eines Sensorwertes \texttt{val} in eine physikalische Einheit hängt von dem gewählten Messbereich ab. Für Beschleunigungswerte gilt
\begin{align*}
	a(\mathtt{val})
		=
			\begin{cases}
				\frac{\mathtt{val}}{16\,384} g & \tn{wenn Messbereich } \pm 2\,g \\
				\frac{\mathtt{val}}{8\,192} g & \tn{wenn Messbereich } \pm 4\,g \\
				\frac{\mathtt{val}}{4\,096} g & \tn{wenn Messbereich } \pm 8\,g \\
				\frac{\mathtt{val}}{2\,048} g & \tn{wenn Messbereich } \pm 16\,g 
			\end{cases}
\end{align*}
und für Drehratenwerte gilt
\begin{align*}
	\omega(\mathtt{val})
		=
			\begin{cases}
				\frac{\mathtt{val}}{131} \mrm{\degree / s} & \tn{wenn Messbereich } \pm \valunit{250}{\degree / s} \\
				\frac{\mathtt{val}}{65,5} \mrm{\degree / s} & \tn{wenn Messbereich } \pm \valunit{500}{\degree / s} \\
				\frac{\mathtt{val}}{32.8} \mrm{\degree / s} & \tn{wenn Messbereich } \pm \valunit{1000}{\degree / s} \\
				\frac{\mathtt{val}}{16.4} \mrm{\degree / s} & \tn{wenn Messbereich } \pm \valunit{2000}{\degree / s}\;.
			\end{cases}
\end{align*}

Standardmäßig sind die Messbereiche $\pm 4\,g$ und $\pm \valunit{500}{\degree / s}$ eingestellt. Diese können jedoch über entsprechende Befehle geändert werden.


\paragraph{Filterung}

Die IMU verfügt über digitale Filter, die vor der sensorinternen Unterabtastung angewendet werden. (Der genaue Aufbau der Filter ist nicht dokumentiert. Die Angabe über Bandbreite und Verzögerung ("`delay"') lassen aber darauf schließen, dass es sich im Wesentlichen um eine Mittelwertbildung handelt.) Die möglichen Einstellungen, die über die ucboard-Befehle vorgenommen werden können, sind in \tabref{tab:hw:imufilters} aufgeführt. Diese Werte sind den Einträgen in der "`Register Map"'-Dokumentation des MPU-9250 entnommen. 

\textbf{Hinweis:} Es ist zu beachten, dass die Abtastung (das Abfragen des Sensorwerte) seitens des ucboard unabhängig von der in \tabref{tab:hw:imufilters} \bzw im Sensordatenblatt genannten Abtastszeit immer mit \valunit{1}{kHz} erfolgt! Auch deshalb sollten die in \tabref{tab:hw:imufilters} grau geschriebenen Einstellungen nur zu Informationszwecken verwendet und nicht für den Normalbetrieb vorgesehen werden.

Die Standardeinstellung der Filter ist jeweils \verb|0|.

\begin{table}%
	\centering
	\caption{Filtereinstellungen IMU}
	\label{tab:hw:imufilters}
	\subfloat[Gyroskop \label{tab:hw:imufilters:gyro}]
	{%
		\begin{tabular}{cccc}
			\mytoprule
			\texttt{GFILT} & Bandbreite [Hz] & Verzögerung [ms] &  Interne Sensor- \\
			& & & abtastzeit [kHz] \\
			\mymidrule
			\color[rgb]{0.5,0.5,0.5}{\texttt{-2}} & \color[rgb]{0.5,0.5,0.5}{8\,800}   & \color[rgb]{0.5,0.5,0.5}{0,064} & \color[rgb]{0.5,0.5,0.5}{32}\\
			\color[rgb]{0.5,0.5,0.5}{\texttt{-1}} & \color[rgb]{0.5,0.5,0.5}{3\,600}   & \color[rgb]{0.5,0.5,0.5}{0,11} & \color[rgb]{0.5,0.5,0.5}{32} \\
			\texttt{0} & 250 & 0,97 & 8 \\
			\texttt{1} & 184 & 2,9 & 1 \\
			\texttt{2} & 92   & 3,9 & 1 \\
			\texttt{3} & 41  & 5,9 & 1 \\
			\texttt{4} & 20  & 9,9 & 1 \\
			\texttt{5} & 10  & 17,85 & 1 \\
			\texttt{6} & 5  & 33,38 & 1 \\
			\color[rgb]{0.5,0.5,0.5}{\texttt{7}} & \color[rgb]{0.5,0.5,0.5}{3\,600}  & \color[rgb]{0.5,0.5,0.5}{0,17} & \color[rgb]{0.5,0.5,0.5}{8} \\
			\mybottomrule
		\end{tabular}%
		}
	\\[10mm]
	\subfloat[Beschleunigungssensor \label{tab:hw:imufilters:acc}]
	{%
		\begin{tabular}{cccc}
			\mytoprule
			\texttt{AFILT} & Bandbreite [Hz] & Verzögerung [ms] &  Interne Sensor- \\
			& & & abtastzeit [kHz] \\
			\mymidrule
			\color[rgb]{0.5,0.5,0.5}{\texttt{-1}} & \color[rgb]{0.5,0.5,0.5}{1\,046} & \color[rgb]{0.5,0.5,0.5}{0,503} & \color[rgb]{0.5,0.5,0.5}{4} \\
			\texttt{0} & 218,1 & 1,88 & 1 \\
			\texttt{1} & 218,1 & 1,88 & 1 \\
			\texttt{2} & 99    & 2,88 & 1 \\
			\texttt{3} & 44,8  & 4,88 & 1 \\
			\texttt{4} & 21,2  & 8,87 & 1 \\
			\texttt{5} & 10,2  & 16,83 & 1 \\
			\texttt{6} & 5,05  & 32,48 & 1 \\
			\texttt{7} & 420   & 1,38 & 1 \\
			\mybottomrule
			\multicolumn{3}{p{8cm}}{\color[rgb]{1,0,0} \tiny Die Einstellungen \texttt{0} und \texttt{1} sind in der Dokumentation tatsächlich identisch angegeben. Das könnte man mal überprüfen.}
		\end{tabular}
		}
\end{table}



\subsection{Magnetometer}

Das Magnetometer vom Typ AK8963C ist im Gehäuse der IMU integriert. (Prinzipiell empfiehlt es sich, direkt das Datenblatt des AK8963 zu verwenden, und nicht in die entsprechenden Passagen des Datenblattes der IMU zu schauen.)

Die Achsenanordnung des Magnetometers unterscheidet sich von der der IMU \bzw damit auch der Achsenanordnung des Fahrzeugs. So zeigt die $x$-Achse zeigt nach links, die $y$-Achse nach vorne und die $z$-Achse nach unten. \textbf{\emph{Alle} ausgegebenen Werte werden jedoch in Fahrzeug- oder IMU-Koordinaten angegeben!}

Es liegt \ca alle \valunit{10}{ms} neue Messwerte vor. (Ungefähr, da die Taktrate des Mikrocontrollers und die des Sensors leicht abweichen können.)


\paragraph{Korrektur der Sensorempfindlichkeit}

Im Magnetometer ist intern für jede Achse ein Korrekturwert für die Sensitivität gespeichert (\texttt{ASAX}, \texttt{ASAY} und \texttt{ASAZ}). Dieser wird beim Starten ausgelesen und zur Korrektur der Messwerte verwendet. Diese Korrektur kann auch abgeschaltet werden (\verb|!MAG OPT ~USEASA=0|).

Der korrigierte Wert $H'_{\{\mrm{x,y,z}\}}$ ergibt sich aus dem Sensorwert $H_{\{\mrm{x,y,z}\}}$ und dem jeweiligen Korrekturfaktor \texttt{ASA}\{\texttt{X},\texttt{Y},\texttt{Z}\} über
\begin{align*}
	H'_{\{\mrm{x,y,z}\}} = H_{\{\mrm{x,y,z}\}} \cdot \left( \frac{(\mathtt{ASA}\{\mathtt{X},\mathtt{Y},\mathtt{Z}\} - 128) \cdot 0,5}{128} + 1 \right)\;.
\end{align*}


\paragraph{Kalibrierung des eingebauten Sensors}

Eine weitere Kalibrierung des Magnetometers ist für die Zukunft vorgesehen. Aktuell muss dies aber, wenn gewünscht \bzw notwendig, selber auf dem PC realisiert werden.


\paragraph{Berechnung des Kurswinkels}

Unter der Annahme, dass das Fahrzeug eben steht, kann aus dem Messwert der magnetischen Feldstärke in $x$- und $y$-Richtung der im(!) Uhrzeigersinn gezählte Kurswinkel $\psi_\mrm{h}$ über
\begin{align*}
	\psi_\mrm{h} = \mrm{atan2}(\mathtt{MY},\ \mathtt{MX})
\end{align*}
in Radiant bestimmt werden. Die Funktion atan2 gibt einen Wert zwischen $-\uppi$ und $\uppi$ zurück. Einen typischen "`Kompasswert"' in Grad erhält man daher nach dem Pseudocode
\begin{verbatimtab}[4]
	heading = atan2(MY, MX) * 180 / pi
	
	if heading < 0
		heading = 360 + heading
	end
\end{verbatimtab}






\FloatBarrier

\clearpage

\chapter{Kommunikation}


\section{Prinzip}


\begin{itemize}
	\item Die Kommunikation des ucboard soll so aufgebaut sein, dass eine Bedienung über ein einfaches Terminalprogramm möglich ist, um ein einfaches Debuggen zu ermöglichen.
	\item Es soll eine einfache Interaktion möglich sein. (\ZB zum Kalibrieren der Sensoren.)
	\item Ein einfaches Parsen der vom ucboard gesendeten Daten soll aber ebenso möglich sein.
\end{itemize}


\begin{itemize}
	\item Textbasiert
		\begin{itemize}
			\item Um bei den Messdaten etwas platzeffizienter zu sein, können diese optional als base64 versendet werden.
		\end{itemize}
	\item Keine \bzw nur bei manchen Befehlen optionale Prüfsummen.
	\item Messdaten und Textnachrichten können vom ucboard ohne Aufforderung versendet werden.
	\item Ansonsten reagiert das ucboard auf Befehle.
\end{itemize}

\begin{itemize}
	\item Die Nachrichten zum ucboard sollen sich möglichst einfach Parsen lassen. Die Nachrichten bestehen aus einem oder mehreren, durch Leerzeichen getrennte Wörter. Der Typ eines Wortes ergibt sich aus dem ersten Zeichen:
		\begin{itemize}
			\item \texttt{A} - \texttt{Z} und \texttt{\_}: String
			\item \verb+~+: Optionsname
			\item \texttt{0} - \texttt{9} und \texttt{-}: Zahl
		\end{itemize}
\end{itemize}



\section{Prinzpielles}

\begin{itemize}
	\item Alle besonderen Startzeichen dürfen auch innerhalb der Nachrichten verwendet werden. Dort haben sie keine besondere Bedeutung.
	\item Besteht eine Nachricht aus mehreren Wörtern, so spielt die Anzahl der Leerzeichen zwischen den Wörtern keine Rolle. (Mindestens eines!)
	\item Nach dem Startzeichen kann, muss aber kein Leerzeichen folgen.
\end{itemize}


\subsubsection{Zu ucboard}
\begin{itemize}
	\item Befehle beginnen mit \texttt{!}
	\item Abfragen beginnen mit \texttt{?}
	\item Antworten auf Fragen des ucboard (im interaktiven Modus) besitzen keinen speziellen Beginn. (Sie dürfen aber auch mit \texttt{!} oder \texttt{?} beginnen.
	\item Das Ende einer Nachricht wird ausschließlich durch \texttt{\textbackslash n} (newline) markiert
	\item Optionen beginnen mit \verb+~+. Wenn der Option ein Wert zugeordnet wird, dann wird dieser nach einem \texttt{=} angeschlossen. Dabei dürfen um das Gleichheitszeichen herum keine Leerzeichen stehen!
\end{itemize}


\subsubsection{Von ucboard}
\begin{itemize}
	\item Direkte Antworten beginnen mit \texttt{:}
	\item Auch jeder Schreibbefehl sollte eine kurze Antwort zur Quittierung senden, \zB \verb+:ok\n+. Es wäre empfehlenswert, den gesetzten Wert zu wiederholen
	\item Fehler bei der Abarbeitung von Befehlen beginnen mit \texttt{:ERR(code)}, wobei \texttt{code} eine positive Ganzzahl als Fehlercode ist. Optional kann nach einem weiteren Doppelpunkt eine Beschreibung des Fehlers folgen: \texttt{:ERR(code):Beschreibung} 
	\item Textausgaben (Display-Funktion) beginnen mit \texttt{'}
	\item Fehlermeldungen sind Textausgaben. Diese beginnen mit \texttt{'ERR(code)}, wobei \texttt{code} eine positive Ganzzahl als Fehlercode ist. Optional kann nach einem weiteren Doppelpunkt eine Beschreibung des Fehlers folgen: \texttt{'ERR(code):Beschreibung} 
	\item Ohne Auf"|forderung versendete Messdaten beginnen mit \texttt{\#} (base64-codiert) \bzw \texttt{\#\#} (lesbarer Text).
	\item Wenn eine Nutzerinteraktion notwendig ist, dann sollte das letzte Zeichen vor dem Nachrichtendezeichen ein \texttt{?} sein.
	\item Alle Nachrichten vom ucboard haben EOT (0x??) als Endzeichen. (Dadurch ist es möglich, eine mehrzeilige Ausgabe auf dem Terminalprogramm zu erzeugen.)
\end{itemize}


\section{Befehle}

\subsubsection{ID}

Car-ID

Abfrage 

\begin{verbatim}
	?ID
	:4
\end{verbatim}


\subsubsection{SID}

Session-ID

\begin{verbatim}
	?SID
	:0
\end{verbatim}


\begin{verbatim}
	!SID 25363
	:25363
\end{verbatim}

\begin{verbatim}
	?SID
	:25363
\end{verbatim}


\subsubsection{STEER}

\begin{verbatim}
	!STEER -200
	:-200
\end{verbatim}

Optionale Wiederholung des Arguments, um Fehlübertragungen zu detektieren

\begin{verbatim}
	!STEER -200 -200
	:-200
\end{verbatim}

\begin{verbatim}
	!STEER -200 -201
	:ERR(1001):Msg corrupted
\end{verbatim}

\begin{verbatim}
	?STEER
	:-200
\end{verbatim}


\subsubsection{DRV}

\begin{verbatim}
	!DRV 300
	:300
\end{verbatim}


Optionale Wiederholung des Arguments, um Fehlübertragungen zu detektieren


\begin{verbatim}
	!DRV 300 300
	:300
\end{verbatim}

\begin{verbatim}
	?DRV
	:300
\end{verbatim}


DRV kann auch ohne Parameter aufgerufen werden. Dann dient er nur dazu, die Zeitzählung der Totmannschaltung (\bzw Tot-PC-Schaltung) neu zu starten.
\begin{verbatim}
	!DRV
	:ok
\end{verbatim}


\paragraph{DMS}

"`Totmannschaltung"' (Dead-man switch)

Dieser Parameter gibt die Zeit in Millisekunden an, innerhalb derer eine neue DRV-Nachricht erhalten sein muss. Wird diese Zeit ohne DRV-Nachricht überschritten, so wird der Motor gestoppt. (Bei der nächsten DRV-Nachricht wird der Motor dann wieder angesteuert.)

\begin{verbatim}
	!DRV ~DMS=1000
	:1000
\end{verbatim}

$-1$ bedeutet deaktiviert ("`unendlich"').


\subsubsection{DAQ}


Signale:\\
\begin{tabular}{lllll}
	\mytoprule
	Signal & Beschreibung & Einheit & Datentyp & Länge \\
	\mymidrule
	\verb|AX| & Beschleunigung $x$-Richtung &  & \verb|int16_t| & 2 \\
	\verb|AY| & Beschleunigung $y$-Richtung &  & \verb|int16_t| & 2 \\
	\verb|AZ| & Beschleunigung $z$-Richtung &  & \verb|int16_t| & 2 \\
	\verb|GX| & Drehrate um $x$-Achse & & \verb|int16_t| & 2 \\
	\verb|GY| & Drehrate um $y$-Achse & & \verb|int16_t| & 2 \\
	\verb|GZ| & Drehrate um $z$-Achse & & \verb|int16_t| & 2 \\
	\verb|USF| & Abstand vorne & mm & \verb|uint16_t| & \\
	\verb|USL| & Abstand links & mm & \verb|uint16_t| & \\
	\verb|USR| & Abstand rechts & mm & \verb|uint16_t| & \\
	\verb|VSBAT| & Spannung Systemakku & mV & \verb|uint16_t| & 2\\
	\verb|VDBAT| & Spannung Fahrakku & mV & \verb|uint16_t| & 2\\
	\mybottomrule
\end{tabular}

Allgemein gilt:

\begin{tabular}{lll}
	\mytoprule
	Datentyp & \verb|uint16_t| & \verb|int16_t| \\
	\mymidrule
	Keine Daten vorhanden & \verb|0xFFFF| & \verb|0x7FFF| \\
	Messfehler & \verb|0xFFFE| & \verb|0x7FFE| \\
	Sensorfehler & \verb|0xFFFD| & \verb|0x7FFD| \\
	Wert zu groß  & \verb|0xFFFC| & \verb|0x7FFC| \\
	Wert zu klein & & \verb|0x8000| \\
	\mybottomrule
\end{tabular}

\begin{tabular}{lll}
	\mytoprule
	Datentyp & \verb|uint16_t| & \verb|int16_t| \\
	\mymidrule
	Keine Daten vorhanden & \verb|[]| & \verb|[]| \\
	Messfehler & \verb|undef| & \verb|undef| \\
	Sensorfehler & \verb|fault| & \verb|fault| \\
	Wert zu groß  & \verb|over| & \verb|over| \\
	Wert zu klein & & \verb|under| \\
	\mybottomrule
\end{tabular}


Einzelabfrage von Werten:
\begin{verbatim}
	!DAQ GET USF
	:453 20
\end{verbatim}


\begin{verbatim}
	!DAQ GET ~AGE USF
	:453 20
\end{verbatim}

Erster Rückgabewert ist Wert (Einheit siehe Tabelle), zweiter Wert ist Alter in Tics

\begin{verbatim}
	!DAQ GET ~AGE USF USL USR
	:453 20 1004 30 323 15
\end{verbatim}



\begin{verbatim}
	!DAQ GET ~TICS USF
	:453 14533
\end{verbatim}


\paragraph{Automatische Messgruppen}

Es stehen zehn automatisierbare Messgruppen zur Verfügung. In diesen können verschiedene Sensorwerte zusammengefasst werden.

Paket 1 enthält die Werte des Ultraschalls. Es wird gesendet, wenn alle Ultraschallwerte vorliegen, wobei maximal \valunit{100}{ms} nach dem Erfassen des ersten Wertes gewartet wird. Die Daten werden base64-codiert und mit einer CRC16-Prüfsumme verschickt.
\begin{verbatim}
	!DAQ MKPKG 1 ~ALL=100 ~B64 ~CRC16 _TIC USF USL USR 
\end{verbatim}

Paket 2 enthält die Spannungen der beiden Akkus, wobei immer eine Nachricht verschickt wird, sobald eine neue Spannung gemessen ist.
\begin{verbatim}
	!DAQ MKPKG 2 ~ANY VSBAT VDBAT 
\end{verbatim}

Paket 3 enthält die Beschleunigungswerte in der Ebene mit der Gierrate. Die Daten werden alle \valunit{10}{ms} verschickt. Dabei werden alle innerhalb der \valunit{10}{ms} erfassten Daten gemittelt.
\begin{verbatim}
	!DAQ MKPKG 3 ~TS=10 ~AVG ~B64 ~CRC16 AX AY GZ 
\end{verbatim}


Optionen:
\begin{itemize}
	\item \verb+~ALL+: Sendet, wenn alle Daten vorhanden sind
	\item \verb+~ANY+: Sendet, wenn ein Datum vorhanden ist
	\item \verb+~B64+: Nachricht wird base64-codiert (ohne Padding) 
	\item \verb+~CRC16+: CRC16-Prüfsumme
	\item \verb+~TS+: Abtastzeit
	\item \verb+~AVG+: Mittelung, wenn TS größer als Sensorabtastzeit ist. (TS muss ein ganzes Vielfaches der Sensorabtastzeit sein.)
\end{itemize}



\begin{tabular}{lllll}
	\mytoprule
	Signal & Beschreibung & & Datentyp & Länge \\
	\mymidrule
	\verb|_TIC| & ucboard-Zeit (tics) der Erfassung des ersten Datums & tics & \verb|uint32_t| & 4 \\
	\verb|_STIC| & die niederwertigen \valunit{16}{bits} von \verb|_TIC| & tics & \verb|uint16_t| & 2 \\
	\verb|_SSTIC| & die niederwertigen \valunit{8}{bits} von \verb|_TIC| & tics & \verb|uint8_t| & 1 \\
	\verb|_DTICS| & Delta der ucboard-Zeit zwischen ersten und letztem Datum & & \verb|int16_t| & 2 \\
	\mybottomrule
\end{tabular}



\begin{verbatim}
	!DAQ SHOWPKGS
	:[...]
\end{verbatim}

\begin{verbatim}
	!DAQ SHOWPKG 1
	:[...]
\end{verbatim}

\begin{verbatim}
	!DAQ RMPKG 2
	:ok
\end{verbatim}


\begin{verbatim}
	!DAQ ACTPKG 1
	:ok
\end{verbatim}

\begin{verbatim}
	!DAQ DEACTPKG 1
	:ok
\end{verbatim}

\begin{verbatim}
	!DAQ START
	:ok
\end{verbatim}

\begin{verbatim}
	!DAQ STOP
	:ok
\end{verbatim}



\subsubsection{IMU CAL}







\FloatBarrier


\part{Entwickeln und Anpassen}

\clearpage


\chapter{Git-Repository}


\section{Ort}

\href{https://github.com/tud-pses/ucboard}{\color[rgb]{0,0,1}https://github.com/tud-pses/ucboard}

\section{Inhalt}

\begin{tabular}{lp{12cm}}
	\verb|datasheets\| & Datenblätter der Sensoren sowie Datenblätter der Bauteile der Schaltungen \\
	\verb|doc\| & Latex-Dateien um dieses Dokument zu erstellen \\
	\verb|fw_releases\| & (firmware\_releases) bin-Dateien der wesentlichen Firmwareversionen \\
	\verb|fw_workspace\| & (firmware\_workspace) Workspace der Entwicklungsumgebung, Sourcecode des auf dem Mikrocontroller laufenden Programms \\
	\verb|kicad_drvbatswitch\| & KiCad-Dateien (Schaltplan und Layout) der kleinen Schaltplatine für den den Fahrakku \\
	\verb|kicad_ucboard\| & KiCad-Dateien (Schaltplan und Layout) der ucboard-Platine \\
	\verb|kicadlibs\| & KiCad-Bibliotheken mit Bauteilen für drvbatswitch und ucboard \\
	\verb|mfg\| & Fertigungsdaten (Gerber-Format) für Platinen \\
	\verb|stm32cubemx\| & Projekt für STM32CubeMX (Programm zur Konfiguration des Mikrocontollers) \\
	\verb|ucterm\| & Einfaches in C\# geschriebenes Terminalprogramm für Entwicklung am ucboard
\end{tabular}


\section{Firmware-Releases}

Im Verzeichnis \verb|fw_releases\| befinden sich die Firmware-Versionen (Programme für den Mikrocontroller) zu bestimmten Versionständen. Die bin-Dateien sind dabei nach dem Schema
\begin{center}
	\texttt{pses\_ucboard\_ver\textit{VERSIONSNUMMER}\_\texttt{BUILDCONF}.bin}
\end{center}
benannt, so \zB
\begin{center}
	\texttt{pses\_ucboard\_ver0.9.0\_Debug.bin}
\end{center}

Zu jeder Version, die in diesem Verzeichnis liegt, sollte im Git-Repository ein Tag der Form
\begin{center}
	\texttt{fwver\_\textit{VERSIONSNUMMER}}
\end{center}
vorhanden sein. Zu dem Beispiel oben gehört also der Tag
\begin{center}
	\texttt{fwver\_0.9.0}
\end{center}

Während der Weiterentwicklung bis zu einem Stand, der eine neue Versionsnummer erhält, sollte der Versionsnummer in \verb|version.h| ein "`\verb|+|"' angehängt werden, also \zB \verb|0.9.0+|. (Idealerweise wird das \verb|+| direkt nach dem Erzeugen der Release-Version angehängt, um zu vermeiden dies zu vergessen.)

Für Weiterentwicklungen innerhalb der Gruppen sollte dem Versionsstring noch der Gruppenname oder eine ähnliche Kennzeichnung hinzugefügt werden.


\tabref{tab:repo:fw_versions} gibt eine Übersicht über die Firmware-Versionen.

\begin{table}%
	\centering
	\caption{Übersicht über Firmware-Versionen}
	\label{tab:repo:fw_versions}
	\begin{tabular}{lrp{10cm}}
		\mytoprule
		Version & Datum & Kommentar \\
		\mymidrule
		0.9.1
			& 01.11.2016
			& -- Befehl \verb|US| hinzugefügt (Ein- und Ausschalten und Parametrierung US-Sensoren) \newline
				-- Beseitigung von Bugs im Kommunikationsstack \\
		0.9.0
			& 25.10.2016
			& Grundfunktionalität weitgehend vorhanden. (Es fehlen noch die Befehle zum Parametrieren und Kalibrieren der Sensoren und der Treiber für das Magnetometer.) \\
		\mybottomrule
	\end{tabular}
\end{table}

\FloatBarrier

\clearpage


\chapter{Flashen}


\section{Linux}

\subsection{Vorbereitung}

STM stellt für den ST-Link/V2-Programmer keine Software für Linux zur Verfügung. Dieser Adapter kann aber dennoch zum Flashen über Linux benutzt werden, wenn das auf GitHub verfügbare Programm \texttt{stlink} verwendet wird. Dieses kann wie folgt installiert werden.\footnote{Vorgehen nach Nicloas Acero.}

\begin{enumerate}
	\item Für Installation benötigte Pakete installieren
		\begin{verbatim}
			sudo apt-get install libusb-1.0-0-dev git unzip libgtk-3-dev
		\end{verbatim}
	\item stlink herunterladen (hier Version 1.2.0, \ggf neuere Version nehmen)
		\begin{verbatim}
			wget https://github.com/texane/stlink/archive/1.2.0.zip
		\end{verbatim}
	\item Datei entpacken und in das Verzeichnis wechseln
		\begin{verbatim}
			# Unzip the file
			unzip 1.2.0.zip
			cd stlink-1.2.0
		\end{verbatim}
	\item Programm kompilieren
		\begin{verbatim}
			mkdir build && cd build
			cmake -DCMAKE_BUILD_TYPE=Debug ..
			make		
		\end{verbatim}
	\item stlink installieren
		\begin{verbatim}
			sudo make install
		\end{verbatim}
	\item udev-Regeln so einstellen, dass st-flash ohne "`sudo"' verwendet werden kann
		\begin{verbatim}
			cd ..
			sudo cp *.rules /etc/udev/rules.d
			sudo restart udev
		\end{verbatim}
	\item Prorgamm sollte sich jetzt öffnen lassen
		\begin{verbatim}
			stlink-gui
		\end{verbatim}
\end{enumerate}


\subsection{Flashen}


\begin{enumerate}
	\item ST-Link/V2 an Platine anschließen. Fahrzeug einschalten.
	\item Öffnen der grafischen Oberfläche von \texttt{stlink}
		\begin{verbatim}
			stlink-gui
		\end{verbatim}
	\item Auf "`Connect"'-Icon klicken. Es sollten die Daten des Mikrocontrollers angezeigt werden.
	\item Auf "`Open File"'-Icon klicken und bin-Datei auswählen.
		\begin{itemize}
			\item Im Repository unter \verb|fw_releases\| liegen die "`offiziellen"' Versionen.
		\end{itemize}
	\item Auf "`Program"'-Icon klicken.
		\begin{itemize}
			\item Start-Adresse ist ist \verb|0x08000000|. (Sollte voreingestellt sein.)
		\end{itemize}
\end{enumerate}


\section{Windows}

\subsection{Vorbereitung}

"`STM32 ST-Link Utility"' von der STM-Homepage herunterladen und installieren (beinhaltet auch die Treiber). 

\textbf{Hinweis:} Die könnte schon problematisch sein, wenn man den ST-Link/V2 zum Debuggen verwendet will, da damit auch ein spezieller Treiber installiert wird, für gbd aber ein generischer Treiber benötigt wird. (Quelle: Forum, quergelesen. Müsste noch geklärt werden.)



\subsection{Flashen}

\begin{enumerate}
	\item ST-Link/V2 an Platine anschließen. Fahrzeug einschalten.
	\item "`STM32 ST-Link/V2 Utility"' öffnen.
	\item Auf "`Connect"'-Icon (drittes von links) klicken. Es sollten die Daten des Mikrocontrollers angezeigt werden.
	\item Auf "`Open File"'-Icon (erstes von links) klicken und bin-Datei auswählen.
		\begin{itemize}
			\item Im Repository unter \verb|fw_releases\| liegen die "`offiziellen"' Versionen.
		\end{itemize}
	\item Auf "`Program Verify"'-Icon (sechstes von links) klicken.
		\begin{itemize}
			\item Start-Adresse ist ist \verb|0x08000000|. (Sollte voreingestellt sein.)
		\end{itemize}
\end{enumerate}




\FloatBarrier

\clearpage


\chapter{Programmierung}


\section{Einrichtung Entwicklungsumgebung}


\paragraph{System Workbench for STM32}

Die verwendete Entwicklungsumgebung "`System Workbench for STM32"' basiert auf Eclipse. Die Entwicklungsumgebung kann über \href{www.openstm32.org}{\color[rgb]{0,0,1} www.openstm32.org} geladen werden. Die Entwicklungsumgebung beinhaltet einen Compiler. 


\paragraph{Projekt importieren}

Wenn das git-Repostory lokal geclont wurde, muss das Projekt einmalig in die Entwicklungsumgebung importiert werden. Dazu wird wie folgt vorgegangen:
\begin{enumerate}
	\item Starten von "`System Workbench for STM32"'
	\item "`File"' $\to$ "`Switch Workspace"' $\to$ "`Other\ldots"': Verzeichnis \verb|fw_workspace\| des Repositories auswählen 
	\item "`File"' $\to$ "`Import"'\ldots:
		\begin{itemize}
			\item "`General"' $\to$ "`Existing Projects into Workspace"'
			\item "`Select root directory"': Verzeichnis \verb|fw_workspace\| des Repositories über "`Browse\ldots"' auswählen 
			\item "`Projects"': pses\_ucboard auswählen
			\item "`Options"': "`Copy projects into workspace"' darf NICHT ausgewählt sein. (Die Daten sind ja schon da, wo sie hingehören.)
			\item "`Finish"'
		\end{itemize}
	\item STRG+B sollte Projekt erstellen
	\item Falls im "`Problems"'-Reiter ein Fehler angezeigt wird, dass ein Symbol nicht aufgelöst werden kann:
		\begin{itemize}
			\item Rechte Maustaste auf Projekt (ersten Eintrag) im "`Project Explorer"', $\to$ "`Properties"'
			\item "`C/C++ General"' $\to$ "`Indexer"'\\ (Falls es den Punkt "`C/C++ General"' nicht gibt, dann hat man vorher nicht auf das Projekt geklickt.)
			\item "`Enable project specific settings"' auswählen
			\item "`Index source files not included in the build"' abwählen
		\end{itemize}
		(Die Ursache liegt darin, dass die Headerdateien für viele Mikrocontrollervarianten vorhanden sind, die alle die gleichen Symbole definieren. Letztlich wird nur ein Header eingebunden, aber mit der genannten Option schaut sich Eclipse dennoch alles an und weiß dann nicht, welches Symbol das richtige ist.)
\end{enumerate}



\paragraph{Einrichten des Programmieradapters ST-Link/V2}

\textbf{Hinweis:} Es müsste möglich sein, den ST-Link/V2 auch zum Debuggen einzurichten. Dies wurde bisher jedoch noch nicht gemacht, sondern er wird lediglich zum Flashen verwendet.

\textbf{Hinweis:} Hier wird die Einrichtung unter Windows beschrieben.

\begin{enumerate}
	\item "`STM32 ST-Link Utility"' von der STM-Homepage herunterladen und installieren (beinhaltet auch die Treiber). \\ \textbf{Hinweis:} Dieser Schritt könnte schon problematisch sein, wenn man den ST-Link/V2 zum Debuggen verwendet will, da damit auch ein spezieller Treiber installiert wird, für gbd aber ein generischer Treiber benötigt wird. (Quelle: Forum, quergelesen. Müsste noch geklärt werden.)
	\item Im Verzeichnis \verb|fw_workspace\pses_ucboard| die Datei \verb|stlinkflash_template.bat| in \verb|stlinkflash.bat| kopieren. (Letztere sollte von Git ignoriert werden.)
	\item In der Datei \verb|stlinkflash.bat| die Pfade zum ST-Link-Utility anpassen.
	\item In der Entwicklungsumgebung: 
		\begin{enumerate}
			\item "`Run"' $\to$ "`External Tools"' $\to$ "`External Tools Configurations\ldots"'
			\item Links auf "`Program"' klicken, dann oben auf das linke Icon zum Erstellen einer neuen Konfiguration
				\begin{itemize}
					\item "`Name"': "`STM flashen"'
					\item "`Location"': Über "`Browse Workspace\ldots"' die Datei \verb|stlinkflash.bat| auswählen
					\item "`Arguments"': \\
						{\small \verb|"${workspace_loc}\${project_name}\${config_name:${project_name}}\${project_name}.bin"|}\\
						Die doppelten Anführungszeichen müssen miteingegeben werden!
				\end{itemize}
			\item "`Apply"'
			\item "`Close"'
		\end{enumerate}
	\item Beim ersten Klicken auf das "`External-Tools"'-Icon (Play-Symbol mit Werkzeugkoffer) kann dann die Konfiguration "`STM flashen"' ausgewählt werden. Danach ist diese automatisch voreingestellt.\\(Die Batch-Datei verwendet das Kommandozeilenprogramm des "`ST-Link-Utilities, um den Chip zu flashen, zu überprüfen und danach zu reseten.)
	\item Sollte sich bei Klick auf das Icon eine Message-Box mit dem Fehler "`Variable references empty selection: \$\{project\_name\}"' erscheinen, dann muss einfach "`in"' eine Datei im Editor-Fenster geklickt werden, so dass der Eingabefokus auf dieser Datei liegt. Wenn dann wieder auf das Icon geklickt wird, sollte es funktionieren.
\end{enumerate}


\paragraph{Weitere Einstellungen}

Standardmäßig speichert die IDE die geänderten Dateien vor einem neuen Build nicht. Um dies zu verändern kann wie folgt vorgegangen werden:
\begin{enumerate}
	\item "`Window"' $\to$ "`Preferences"'
	\item In der linken Spalte "`General"' $\to$ "`Workspace"' auswählen.
	\item Die Option "`Save automatically before build"' auswählen.
\end{enumerate}


\section{Übersicht über Sourcecode}


\section{Belegung der Ressourcen}



\FloatBarrier

\clearpage

\chapter{Schaltung}




\section{Auslegung}



\subsubsection{Spannungsteiler mit RC-Glied}

\begin{figure}[htb]%
	\centering
	\begin{tikzpicture}
		\ctikzset {label/align = straight}
		\draw[color=black, thick]
			(0,0) to [short,o-] (1,0)
				to [R,-*,l^=$R_1$] (3,0)
				to [R,-*,l^=$R_3$] (5,0)
				to [short,-o] (7,0)
			(3,0) to [R,-*,l^=$R_2$] (3,-2)
			(5,0) to [C,-*,l^=$C$] (5,-2)
			(0,-2) to [short,o-o] (7,-2);
			
		\draw[->, thick] (0,-0.2) -- node [auto,swap] {$U_\mrm{in}$} (0,-1.8);
		\draw[->, thick] (7,-0.2) -- node [auto] {$U_\mrm{out}$} (7,-1.8);
	\end{tikzpicture}
	\caption{Spannungsteiler mit RC-Glied}%
	\label{fig:VDivRC}%
\end{figure}		

Die Schaltung aus \figref{fig:VDivRC} (ideale Spannungsquelle am Eingang, offene Klemmen am Ausgang) stellt ein PT$_1$-Glied mit der stationären Verstärkung
\begin{align*}
	\frac{R_2}{R_1 + R_2}
\end{align*}
und der Zeitkonstante
\begin{align*}
	T = C \cdot \left( \frac{R_1 R_2}{R_1 + R_2} + R_3 \right)
\end{align*}
dar.
\FloatBarrier

\clearpage


\chapter{Aufbau Fahrzeug}


\section{Erstinbetriebnahme}

\begin{itemize}
	\item Bevor der Fahrzeugaufbau auf das Chassis gesetzt wird, muss der Fahrtenregler kalibriert werden.
\end{itemize}


\FloatBarrier

% =================================================================================



% =================================================================================
% Anhang
% =================================================================================
%\appendix
%\part{Anhang}


%% =================================================================================
%% Literaturverzeichnis
%% =================================================================================
%\clearpage						% Auf eine leere Seite einfügen
%\phantomsection					% Für Aufnahme ins Inhaltsverzeichnis
%\addcontentsline{toc}{chapter}{\bibname}	% In Inhaltsverzeichnis von
											%% Dokument und pdf aufnehmen
%\bibliographystyle{gerabbrv}	% Festlegen, wie das Verzeichnis und die Verweise
								%% im Text aussehen
%\bibliography{D:/Literatur/literature_complete}
								%% Literaturverzeichnis einfügen, mit Angabe der 
								%% Bibtex-Datei
%% =================================================================================



% =================================================================================
% Abbildungsverzeichnis
% =================================================================================
%\clearpage
%\phantomsection					% Für Aufnahme ins Inhaltsverzeichnis
%\addcontentsline{toc}{chapter}{\listfigurename}	% In Inhaltsverzeichnis von
%												% Dokument und pdf aufnehmen
%\listoffigures
% =================================================================================

% =================================================================================
% Tabellenverzeichnis
% =================================================================================
%\clearpage
%\phantomsection					% Für Aufnahme ins Inhaltsverzeichnis
%\addcontentsline{toc}{chapter}{\listtablename}	% In Inhaltsverzeichnis von
%												% Dokument und pdf aufnehmen
%\listoftables
% =================================================================================

\end{document}
